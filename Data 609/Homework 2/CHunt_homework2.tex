\documentclass[]{article}
\usepackage{lmodern}
\usepackage{amssymb,amsmath}
\usepackage{ifxetex,ifluatex}
\usepackage{fixltx2e} % provides \textsubscript
\ifnum 0\ifxetex 1\fi\ifluatex 1\fi=0 % if pdftex
  \usepackage[T1]{fontenc}
  \usepackage[utf8]{inputenc}
\else % if luatex or xelatex
  \ifxetex
    \usepackage{mathspec}
  \else
    \usepackage{fontspec}
  \fi
  \defaultfontfeatures{Ligatures=TeX,Scale=MatchLowercase}
\fi
% use upquote if available, for straight quotes in verbatim environments
\IfFileExists{upquote.sty}{\usepackage{upquote}}{}
% use microtype if available
\IfFileExists{microtype.sty}{%
\usepackage{microtype}
\UseMicrotypeSet[protrusion]{basicmath} % disable protrusion for tt fonts
}{}
\usepackage[margin=1in]{geometry}
\usepackage{hyperref}
\hypersetup{unicode=true,
            pdftitle={Homework 2},
            pdfauthor={Christophe Hunt},
            pdfborder={0 0 0},
            breaklinks=true}
\urlstyle{same}  % don't use monospace font for urls
\usepackage{color}
\usepackage{fancyvrb}
\newcommand{\VerbBar}{|}
\newcommand{\VERB}{\Verb[commandchars=\\\{\}]}
\DefineVerbatimEnvironment{Highlighting}{Verbatim}{commandchars=\\\{\}}
% Add ',fontsize=\small' for more characters per line
\usepackage{framed}
\definecolor{shadecolor}{RGB}{248,248,248}
\newenvironment{Shaded}{\begin{snugshade}}{\end{snugshade}}
\newcommand{\KeywordTok}[1]{\textcolor[rgb]{0.13,0.29,0.53}{\textbf{{#1}}}}
\newcommand{\DataTypeTok}[1]{\textcolor[rgb]{0.13,0.29,0.53}{{#1}}}
\newcommand{\DecValTok}[1]{\textcolor[rgb]{0.00,0.00,0.81}{{#1}}}
\newcommand{\BaseNTok}[1]{\textcolor[rgb]{0.00,0.00,0.81}{{#1}}}
\newcommand{\FloatTok}[1]{\textcolor[rgb]{0.00,0.00,0.81}{{#1}}}
\newcommand{\ConstantTok}[1]{\textcolor[rgb]{0.00,0.00,0.00}{{#1}}}
\newcommand{\CharTok}[1]{\textcolor[rgb]{0.31,0.60,0.02}{{#1}}}
\newcommand{\SpecialCharTok}[1]{\textcolor[rgb]{0.00,0.00,0.00}{{#1}}}
\newcommand{\StringTok}[1]{\textcolor[rgb]{0.31,0.60,0.02}{{#1}}}
\newcommand{\VerbatimStringTok}[1]{\textcolor[rgb]{0.31,0.60,0.02}{{#1}}}
\newcommand{\SpecialStringTok}[1]{\textcolor[rgb]{0.31,0.60,0.02}{{#1}}}
\newcommand{\ImportTok}[1]{{#1}}
\newcommand{\CommentTok}[1]{\textcolor[rgb]{0.56,0.35,0.01}{\textit{{#1}}}}
\newcommand{\DocumentationTok}[1]{\textcolor[rgb]{0.56,0.35,0.01}{\textbf{\textit{{#1}}}}}
\newcommand{\AnnotationTok}[1]{\textcolor[rgb]{0.56,0.35,0.01}{\textbf{\textit{{#1}}}}}
\newcommand{\CommentVarTok}[1]{\textcolor[rgb]{0.56,0.35,0.01}{\textbf{\textit{{#1}}}}}
\newcommand{\OtherTok}[1]{\textcolor[rgb]{0.56,0.35,0.01}{{#1}}}
\newcommand{\FunctionTok}[1]{\textcolor[rgb]{0.00,0.00,0.00}{{#1}}}
\newcommand{\VariableTok}[1]{\textcolor[rgb]{0.00,0.00,0.00}{{#1}}}
\newcommand{\ControlFlowTok}[1]{\textcolor[rgb]{0.13,0.29,0.53}{\textbf{{#1}}}}
\newcommand{\OperatorTok}[1]{\textcolor[rgb]{0.81,0.36,0.00}{\textbf{{#1}}}}
\newcommand{\BuiltInTok}[1]{{#1}}
\newcommand{\ExtensionTok}[1]{{#1}}
\newcommand{\PreprocessorTok}[1]{\textcolor[rgb]{0.56,0.35,0.01}{\textit{{#1}}}}
\newcommand{\AttributeTok}[1]{\textcolor[rgb]{0.77,0.63,0.00}{{#1}}}
\newcommand{\RegionMarkerTok}[1]{{#1}}
\newcommand{\InformationTok}[1]{\textcolor[rgb]{0.56,0.35,0.01}{\textbf{\textit{{#1}}}}}
\newcommand{\WarningTok}[1]{\textcolor[rgb]{0.56,0.35,0.01}{\textbf{\textit{{#1}}}}}
\newcommand{\AlertTok}[1]{\textcolor[rgb]{0.94,0.16,0.16}{{#1}}}
\newcommand{\ErrorTok}[1]{\textcolor[rgb]{0.64,0.00,0.00}{\textbf{{#1}}}}
\newcommand{\NormalTok}[1]{{#1}}
\usepackage{graphicx,grffile}
\makeatletter
\def\maxwidth{\ifdim\Gin@nat@width>\linewidth\linewidth\else\Gin@nat@width\fi}
\def\maxheight{\ifdim\Gin@nat@height>\textheight\textheight\else\Gin@nat@height\fi}
\makeatother
% Scale images if necessary, so that they will not overflow the page
% margins by default, and it is still possible to overwrite the defaults
% using explicit options in \includegraphics[width, height, ...]{}
\setkeys{Gin}{width=\maxwidth,height=\maxheight,keepaspectratio}
\IfFileExists{parskip.sty}{%
\usepackage{parskip}
}{% else
\setlength{\parindent}{0pt}
\setlength{\parskip}{6pt plus 2pt minus 1pt}
}
\setlength{\emergencystretch}{3em}  % prevent overfull lines
\providecommand{\tightlist}{%
  \setlength{\itemsep}{0pt}\setlength{\parskip}{0pt}}
\setcounter{secnumdepth}{5}
% Redefines (sub)paragraphs to behave more like sections
\ifx\paragraph\undefined\else
\let\oldparagraph\paragraph
\renewcommand{\paragraph}[1]{\oldparagraph{#1}\mbox{}}
\fi
\ifx\subparagraph\undefined\else
\let\oldsubparagraph\subparagraph
\renewcommand{\subparagraph}[1]{\oldsubparagraph{#1}\mbox{}}
\fi

%%% Use protect on footnotes to avoid problems with footnotes in titles
\let\rmarkdownfootnote\footnote%
\def\footnote{\protect\rmarkdownfootnote}

%%% Change title format to be more compact
\usepackage{titling}

% Create subtitle command for use in maketitle
\newcommand{\subtitle}[1]{
  \posttitle{
    \begin{center}\large#1\end{center}
    }
}

\setlength{\droptitle}{-2em}
  \title{Homework 2}
  \pretitle{\vspace{\droptitle}\centering\huge}
  \posttitle{\par}
  \author{Christophe Hunt}
  \preauthor{\centering\large\emph}
  \postauthor{\par}
  \predate{\centering\large\emph}
  \postdate{\par}
  \date{February 8, 2017}

\usepackage{relsize}
\usepackage{setspace}
\usepackage{amsmath,amsfonts,amsthm}
\usepackage[sfdefault]{roboto}
\usepackage[T1]{fontenc}
\usepackage{float}

\begin{document}
\maketitle

{
\setcounter{tocdepth}{2}
\tableofcontents
}
\section{Page 69: problem 12}\label{page-69-problem-12}

From this vague scenario, identify a problem you would like to study.
Which variables affect the behavior you have identified in the problem
identification? Which variables are the most important?

A company with a fleet of trucks faces increasing maintenance costs as
the age and mileage of the trucks increase.A problem that would be
interesting to study is the at what point should the truck be retired
and a new vehicle purchased. The costs associated with a new purchase
would need to outweigh the cost of maintaining the aged truck.

The variables of importance would be the maintenance cost as its
associated with the age and mileage of the truck, any additional
variables such as the severity of past repairs. I would assume that a
vehicle with engine failure may have future issues until the engine is
replaced, at which point the cost of maintenance may not be as severe.
Additionally, the cost of a new purchase would be the continuing
payments, depreciation, maintenance, fuel efficiency, and the
opportunity costs of reliability.

The equilibrium of this system would be meaningful to make a data driven
decision to buy a new vehicle or to keep running the aged truck.

\newpage

\section{Page 79: problem 11}\label{page-79-problem-11}

Determine whether the data set supports the stated proportionality
model.

\[y \propto x^3\]

\begin{table}[!htbp]
\centering
\caption{}
\label{my-label}
\begin{tabular}{l|llllllllll}
y & 0 & 1 & 2 & 6 & 14 & 24 & 37 & 58 & 82 & 114 \\ \hline
x & 1 & 2 & 3 & 4 & 5 & 6 & 7 & 8 & 9 & 10
\end{tabular}
\end{table}

\begin{Shaded}
\begin{Highlighting}[]
\KeywordTok{library}\NormalTok{(ggplot2)}
\NormalTok{y <-}\StringTok{ }\KeywordTok{c}\NormalTok{(}\DecValTok{0}\NormalTok{, }\DecValTok{1}\NormalTok{, }\DecValTok{2}\NormalTok{, }\DecValTok{6}\NormalTok{, }\DecValTok{14}\NormalTok{, }\DecValTok{24}\NormalTok{, }\DecValTok{37}\NormalTok{, }\DecValTok{58}\NormalTok{, }\DecValTok{82}\NormalTok{, }\DecValTok{114}\NormalTok{)}
\NormalTok{x <-}\StringTok{ }\KeywordTok{c}\NormalTok{(}\DecValTok{1}\NormalTok{:}\DecValTok{10}\NormalTok{)}
\NormalTok{df <-}\StringTok{ }\KeywordTok{as.data.frame}\NormalTok{(}\KeywordTok{cbind}\NormalTok{(y,x))}
\NormalTok{df$y2 <-}\StringTok{ }\NormalTok{df$y^(}\DecValTok{1}\NormalTok{/}\DecValTok{3}\NormalTok{)}
\NormalTok{df$k <-}\StringTok{ }\NormalTok{df$y2/df$x}
\NormalTok{k <-}\StringTok{ }\KeywordTok{mean}\NormalTok{(df$k)}
\NormalTok{df$model <-}\StringTok{ }\NormalTok{(k^}\DecValTok{3}\NormalTok{)*df$x^}\DecValTok{3}

\KeywordTok{ggplot}\NormalTok{() +}\StringTok{ }
\StringTok{  }\KeywordTok{geom_line}\NormalTok{(}\DataTypeTok{data =} \NormalTok{df, }\KeywordTok{aes}\NormalTok{(x, y), }\DataTypeTok{color =} \StringTok{'red'}\NormalTok{) +}
\StringTok{  }\KeywordTok{geom_line}\NormalTok{(}\DataTypeTok{data =} \NormalTok{df, }\KeywordTok{aes}\NormalTok{(x,model), }\DataTypeTok{color =} \StringTok{'green'}\NormalTok{) +}
\StringTok{  }\KeywordTok{theme_minimal}\NormalTok{() +}
\StringTok{  }\KeywordTok{ylim}\NormalTok{(}\DecValTok{0}\NormalTok{,}\DecValTok{100}\NormalTok{)}
\end{Highlighting}
\end{Shaded}

\includegraphics{CHunt_homework2_files/figure-latex/unnamed-chunk-1-1.pdf}

The data does not support the proportion model since our data does not
pass through the origin (0,0) and our slope is small comparative to 1.
Our used proportional model for this data is \(y = .0.078x^3\) where
\(k = 0.078\). This is achieved by taking \(y^\frac{1}{3}\) and from the
values in our provided data set. Then we obtain the ratio of
\(\frac{y^\frac{1}{3}}{x}\) and further obtain the mean which is
\(0.078\). As illustrated above the model illustrates that the data does
not follow the proportional model.

\section{Page 94: problem 4}\label{page-94-problem-4}

Lumber Cutters - Lumber cutters wish to use readily available
measurements to estimate the number of board feet for lumber in a tree.
Assume they measure the diameter of the tree in inches at waist height.
Develop a model that predicts board feet as a function of diameter in
inches.

Use the following data for your test.

\begin{table}[!htbp]
\centering
\caption{}
\label{my-label}
\begin{tabular}{l|llllllllll}
x & 17 & 19 & 20 & 23 & 25 & 28 & 32 & 38 & 39 & 41 \\ \hline
y & 19 & 25 & 32 & 57 & 71 & 113 & 123 & 252 & 259 & 294
\end{tabular}
\end{table}

\begin{Shaded}
\begin{Highlighting}[]
\NormalTok{x <-}\StringTok{ }\KeywordTok{c}\NormalTok{(}\DecValTok{17}\NormalTok{,}\DecValTok{19}\NormalTok{,}\DecValTok{20}\NormalTok{,}\DecValTok{23}\NormalTok{,}\DecValTok{25}\NormalTok{,}\DecValTok{28}\NormalTok{,}\DecValTok{32}\NormalTok{,}\DecValTok{38}\NormalTok{,}\DecValTok{39}\NormalTok{,}\DecValTok{41}\NormalTok{)}
\NormalTok{y <-}\StringTok{ }\KeywordTok{c}\NormalTok{(}\DecValTok{19}\NormalTok{,}\DecValTok{25}\NormalTok{,}\DecValTok{32}\NormalTok{,}\DecValTok{57}\NormalTok{,}\DecValTok{71}\NormalTok{,}\DecValTok{113}\NormalTok{,}\DecValTok{123}\NormalTok{,}\DecValTok{252}\NormalTok{,}\DecValTok{259}\NormalTok{,}\DecValTok{294}\NormalTok{)}
\NormalTok{df <-}\StringTok{ }\KeywordTok{as.data.frame}\NormalTok{(}\KeywordTok{cbind}\NormalTok{(x,y))}
\end{Highlighting}
\end{Shaded}

The variable \(x\) is the diameter of a ponderous pine in inches, and y
is the number of board feet divided by 10.

\begin{enumerate}
\def\labelenumi{\alph{enumi}.}
\item
  Consider two separate assumptions, allowing each to lead to a model.
  Completely analyze each model.
\item
  Assume that all trees are right-circular cylinders and are
  approximately the same height.
\end{enumerate}

We are assuming proportional change on two dimensions; the diameter
change and the change in right-circular cylinders (excluding height), we
therefore have the proportional model \(y \propto x^2\)

\begin{Shaded}
\begin{Highlighting}[]
\KeywordTok{library}\NormalTok{(ggplot2)}
\NormalTok{df$y2 <-}\StringTok{ }\NormalTok{df$y^(}\DecValTok{1}\NormalTok{/}\DecValTok{2}\NormalTok{)}
\NormalTok{df$k <-}\StringTok{ }\NormalTok{df$y2/df$x}
\NormalTok{k <-}\StringTok{ }\KeywordTok{mean}\NormalTok{(df$k)}
\NormalTok{df$model <-}\StringTok{ }\NormalTok{(k^}\DecValTok{2}\NormalTok{)*df$x^}\DecValTok{2}

\KeywordTok{ggplot}\NormalTok{() +}\StringTok{ }
\StringTok{  }\KeywordTok{geom_line}\NormalTok{(}\DataTypeTok{data =} \NormalTok{df, }\KeywordTok{aes}\NormalTok{(x, y), }\DataTypeTok{color =} \StringTok{'red'}\NormalTok{) +}
\StringTok{  }\KeywordTok{geom_line}\NormalTok{(}\DataTypeTok{data =} \NormalTok{df, }\KeywordTok{aes}\NormalTok{(x, model), }\DataTypeTok{color =} \StringTok{'blue'}\NormalTok{) +}
\StringTok{  }\KeywordTok{theme_minimal}\NormalTok{() }
\end{Highlighting}
\end{Shaded}

\includegraphics{CHunt_homework2_files/figure-latex/unnamed-chunk-3-1.pdf}

\newpage

\begin{enumerate}
\def\labelenumi{\roman{enumi}.}
\setcounter{enumi}{1}
\tightlist
\item
  Assume that all trees are right-circular cylinders and that the height
  of the tree is proportional to the diameter.
\end{enumerate}

We are assuming proportional change on three dimensions; the diameter
change, the change in right-circular cylinders, and the change in
height, we therefore have the proportional model \(y \propto x^3\)

\begin{Shaded}
\begin{Highlighting}[]
\KeywordTok{library}\NormalTok{(ggplot2)}
\NormalTok{df$y2 <-}\StringTok{ }\NormalTok{df$x^(}\DecValTok{1}\NormalTok{/}\DecValTok{3}\NormalTok{)}
\NormalTok{df$k <-}\StringTok{ }\NormalTok{df$y2/df$x}
\NormalTok{k <-}\StringTok{ }\KeywordTok{mean}\NormalTok{(df$k)}
\NormalTok{df$model <-}\StringTok{ }\NormalTok{(k^}\DecValTok{3}\NormalTok{)*df$x^}\DecValTok{3}

\KeywordTok{ggplot}\NormalTok{() +}\StringTok{ }
\StringTok{  }\KeywordTok{geom_line}\NormalTok{(}\DataTypeTok{data =} \NormalTok{df, }\KeywordTok{aes}\NormalTok{(x, y), }\DataTypeTok{color =} \StringTok{'red'}\NormalTok{) +}
\StringTok{  }\KeywordTok{geom_line}\NormalTok{(}\DataTypeTok{data =} \NormalTok{df, }\KeywordTok{aes}\NormalTok{(x,model), }\DataTypeTok{color =} \StringTok{'blue'}\NormalTok{) +}
\StringTok{  }\KeywordTok{theme_minimal}\NormalTok{() }
\end{Highlighting}
\end{Shaded}

\includegraphics{CHunt_homework2_files/figure-latex/unnamed-chunk-4-1.pdf}

\begin{enumerate}
\def\labelenumi{\alph{enumi}.}
\setcounter{enumi}{1}
\tightlist
\item
  Which model appears to be better? Why? Justify your conclusions.
\end{enumerate}

The first model for \(y \propto x^2\) because it more closely follows
the shape of the observed data. It seems that the model
\(y \propto x^3\) does not model the growth as accuractly. This is
likely due to tree growth height is not as proportional to the the
diameter of the tree as one would expect. This does not seem as
intuitive as I would have expected.

\section{Page 99: problem 3}\label{page-99-problem-3}

Discuss several factors that were completely ignored in our analysis of
the gasoline mileage problem.

\begin{quote}
Several factors that were completely ignored are air temperature and the
age of the vehicle. Hybrid and electric vehicles are becoming more
common and newer engines are much more fuel efficient so I believe that
the age of the vehicle would have impact on an appropriate model. The
ambient air temperature has impact on the air pressure of the tires and
the possible friction with the road which would have an impact on our
model. The model limitations statement at the end of the Automobile
Gasoline Mileage is very clear that the model is quite fragile and has
limit application. A limitation statement like this indicates the
application of the model is very narrow.
\end{quote}

\newpage

\section{Page 104: problem 2}\label{page-104-problem-2}

Tests exist to measure the percentage of body fat. Assume that such
tests are accurate and that a great many carefully collected data are
available. You may specify any other statistic, such as waist size and
height, that you would like collected. Explain how the data could be
arranged to check the assumptions underlying the sub models in this
section. For example, suppose the data for males between ages 17 and 21
with constant body fat and height are examined. Explain how the
assumption of constant density of the inner core could be checked.

\begin{quote}
We begin by assuming that for adults certain parts will have the same
density of the inner core for different people.
\(D_{in} = k_1 + H + V_{in} - V_{out}\) where \(D_{in}\) = the density
of the inner core and \(k_1 > 0\) is the constant body fat of body parts
with the same volume and density of other individuals. Also, the
\(V_{out}\) is the outer volume of the core which contains the most body
fat, we assume that body fat is only contained in the outer core and the
inner core is muscle and bone which has different density, hence the
reduction in our model. \(V_{in}\) equals the volume of the inner core.
\end{quote}

\begin{quote}
We begin by utilizing already provided submodels from the text. The
volume of both the inner core and outer core is proportional to the cube
of which we select to be height h. The sum of the components must be
proportional to the cube of the height, or \(V_{in} \propto h^3\). We
can continue to use this submodel as we will assume that body fat has a
proportional relationship to volume of the outer core.
\end{quote}

\begin{quote}
We will assume that body fat is represented in
\(V_{out} = BF{total}~V{out}\) where \(BF{total}\) is the total body fat
of the person and \(V{out}\) equals the volume of the outer core. We
will further assume that this is proportional to height cubed.
Substituting in the values we have the following model:
\end{quote}

\[D_{in} = k_1 + h^3 - k_2h^3~~~for~~~k_1, k_2 > 0\]

\begin{quote}
The model suggests that variation in the denisty of the inner core is
based on the volume of the inner core less the body fat \(k_2\) and the
volume of the outer core.
\end{quote}


\end{document}
