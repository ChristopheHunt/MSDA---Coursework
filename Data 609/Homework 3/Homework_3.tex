\documentclass[]{article}
\usepackage{lmodern}
\usepackage{amssymb,amsmath}
\usepackage{ifxetex,ifluatex}
\usepackage{fixltx2e} % provides \textsubscript
\ifnum 0\ifxetex 1\fi\ifluatex 1\fi=0 % if pdftex
  \usepackage[T1]{fontenc}
  \usepackage[utf8]{inputenc}
\else % if luatex or xelatex
  \ifxetex
    \usepackage{mathspec}
  \else
    \usepackage{fontspec}
  \fi
  \defaultfontfeatures{Ligatures=TeX,Scale=MatchLowercase}
\fi
% use upquote if available, for straight quotes in verbatim environments
\IfFileExists{upquote.sty}{\usepackage{upquote}}{}
% use microtype if available
\IfFileExists{microtype.sty}{%
\usepackage{microtype}
\UseMicrotypeSet[protrusion]{basicmath} % disable protrusion for tt fonts
}{}
\usepackage[margin=1in]{geometry}
\usepackage{hyperref}
\hypersetup{unicode=true,
            pdftitle={Homework 3},
            pdfauthor={Christophe Hunt},
            pdfborder={0 0 0},
            breaklinks=true}
\urlstyle{same}  % don't use monospace font for urls
\usepackage{graphicx,grffile}
\makeatletter
\def\maxwidth{\ifdim\Gin@nat@width>\linewidth\linewidth\else\Gin@nat@width\fi}
\def\maxheight{\ifdim\Gin@nat@height>\textheight\textheight\else\Gin@nat@height\fi}
\makeatother
% Scale images if necessary, so that they will not overflow the page
% margins by default, and it is still possible to overwrite the defaults
% using explicit options in \includegraphics[width, height, ...]{}
\setkeys{Gin}{width=\maxwidth,height=\maxheight,keepaspectratio}
\IfFileExists{parskip.sty}{%
\usepackage{parskip}
}{% else
\setlength{\parindent}{0pt}
\setlength{\parskip}{6pt plus 2pt minus 1pt}
}
\setlength{\emergencystretch}{3em}  % prevent overfull lines
\providecommand{\tightlist}{%
  \setlength{\itemsep}{0pt}\setlength{\parskip}{0pt}}
\setcounter{secnumdepth}{5}
% Redefines (sub)paragraphs to behave more like sections
\ifx\paragraph\undefined\else
\let\oldparagraph\paragraph
\renewcommand{\paragraph}[1]{\oldparagraph{#1}\mbox{}}
\fi
\ifx\subparagraph\undefined\else
\let\oldsubparagraph\subparagraph
\renewcommand{\subparagraph}[1]{\oldsubparagraph{#1}\mbox{}}
\fi

%%% Use protect on footnotes to avoid problems with footnotes in titles
\let\rmarkdownfootnote\footnote%
\def\footnote{\protect\rmarkdownfootnote}

%%% Change title format to be more compact
\usepackage{titling}

% Create subtitle command for use in maketitle
\newcommand{\subtitle}[1]{
  \posttitle{
    \begin{center}\large#1\end{center}
    }
}

\setlength{\droptitle}{-2em}
  \title{Homework 3}
  \pretitle{\vspace{\droptitle}\centering\huge}
  \posttitle{\par}
  \author{Christophe Hunt}
  \preauthor{\centering\large\emph}
  \postauthor{\par}
  \predate{\centering\large\emph}
  \postdate{\par}
  \date{February 18, 2017}

\usepackage{relsize}
\usepackage{setspace}
\usepackage{amsmath,amsfonts,amsthm}
\usepackage[sfdefault]{roboto}
\usepackage[T1]{fontenc}
\usepackage{float}

\begin{document}
\maketitle

{
\setcounter{tocdepth}{2}
\tableofcontents
}
\section{Problem : Page 113: 2}\label{problem-page-113-2}

The following table gives the elongation \(e\) in inches (in./in.) for a
given stress \(S\) on a steel wire measured in pounds per square inch
(lb/in.\(^2\)). Test the models \(e = c_1S\) by plotting the data.
Estimate \(c_1\) graphically.

\begin{table}[!htbp]
\centering
\caption{}
\label{my-label}
\begin{tabular}{l|lllllllllll}
$S(x10^{-3})$ & 5 & 10 & 20 & 30 & 40  & 50  & 60  & 70  & 80  & 90  & 100 \\ \hline
$e(x10^5)$    & 0 & 19 & 57 & 94 & 134 & 173 & 216 & 256 & 297 & 343 & 390 
\end{tabular}
\end{table}

\section{Problem : Page 121: 2.a}\label{problem-page-121-2.a}

For each of the following data sets, formulate the mathematical model
that minimizes the largest deviation between the data and the line y=
ax+b. If a compuet is availae solve for the estiamtes of a and b.

\begin{table}[!htbp]
\centering
\caption{}
\label{my-label}
\begin{tabular}{l|llllll}
x & 1   & 2.3 & 3.7 & 4.2 & 6.1 & 7.0 \\ \hline
y & 3.6 & 3.0 & 3.2 & 5.1 & 5.3 & 6.8  
\end{tabular}
\end{table}

\section{Problem : Page 127: 10}\label{problem-page-127-10}

\section{Problem : Page 136: 7}\label{problem-page-136-7}

\section{Problem : Page 146: 5}\label{problem-page-146-5}

\section{Problem : Page 157: 4}\label{problem-page-157-4}

\section{Problem : Page 169: 11}\label{problem-page-169-11}

\section{Problem : Page 181: 5}\label{problem-page-181-5}


\end{document}
