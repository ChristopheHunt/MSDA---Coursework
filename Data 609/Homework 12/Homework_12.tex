\documentclass[]{article}
\usepackage{lmodern}
\usepackage{amssymb,amsmath}
\usepackage{ifxetex,ifluatex}
\usepackage{fixltx2e} % provides \textsubscript
\ifnum 0\ifxetex 1\fi\ifluatex 1\fi=0 % if pdftex
  \usepackage[T1]{fontenc}
  \usepackage[utf8]{inputenc}
\else % if luatex or xelatex
  \ifxetex
    \usepackage{mathspec}
  \else
    \usepackage{fontspec}
  \fi
  \defaultfontfeatures{Ligatures=TeX,Scale=MatchLowercase}
\fi
% use upquote if available, for straight quotes in verbatim environments
\IfFileExists{upquote.sty}{\usepackage{upquote}}{}
% use microtype if available
\IfFileExists{microtype.sty}{%
\usepackage{microtype}
\UseMicrotypeSet[protrusion]{basicmath} % disable protrusion for tt fonts
}{}
\usepackage[margin=1in]{geometry}
\usepackage{hyperref}
\hypersetup{unicode=true,
            pdftitle={Homework 12},
            pdfauthor={Christophe Hunt},
            pdfborder={0 0 0},
            breaklinks=true}
\urlstyle{same}  % don't use monospace font for urls
\usepackage{graphicx,grffile}
\makeatletter
\def\maxwidth{\ifdim\Gin@nat@width>\linewidth\linewidth\else\Gin@nat@width\fi}
\def\maxheight{\ifdim\Gin@nat@height>\textheight\textheight\else\Gin@nat@height\fi}
\makeatother
% Scale images if necessary, so that they will not overflow the page
% margins by default, and it is still possible to overwrite the defaults
% using explicit options in \includegraphics[width, height, ...]{}
\setkeys{Gin}{width=\maxwidth,height=\maxheight,keepaspectratio}
\IfFileExists{parskip.sty}{%
\usepackage{parskip}
}{% else
\setlength{\parindent}{0pt}
\setlength{\parskip}{6pt plus 2pt minus 1pt}
}
\setlength{\emergencystretch}{3em}  % prevent overfull lines
\providecommand{\tightlist}{%
  \setlength{\itemsep}{0pt}\setlength{\parskip}{0pt}}
\setcounter{secnumdepth}{5}
% Redefines (sub)paragraphs to behave more like sections
\ifx\paragraph\undefined\else
\let\oldparagraph\paragraph
\renewcommand{\paragraph}[1]{\oldparagraph{#1}\mbox{}}
\fi
\ifx\subparagraph\undefined\else
\let\oldsubparagraph\subparagraph
\renewcommand{\subparagraph}[1]{\oldsubparagraph{#1}\mbox{}}
\fi

%%% Use protect on footnotes to avoid problems with footnotes in titles
\let\rmarkdownfootnote\footnote%
\def\footnote{\protect\rmarkdownfootnote}

%%% Change title format to be more compact
\usepackage{titling}

% Create subtitle command for use in maketitle
\newcommand{\subtitle}[1]{
  \posttitle{
    \begin{center}\large#1\end{center}
    }
}

\setlength{\droptitle}{-2em}
  \title{Homework 12}
  \pretitle{\vspace{\droptitle}\centering\huge}
  \posttitle{\par}
  \author{Christophe Hunt}
  \preauthor{\centering\large\emph}
  \postauthor{\par}
  \predate{\centering\large\emph}
  \postdate{\par}
  \date{April 29, 2017}

\usepackage{relsize}
\usepackage{setspace}
\usepackage{amsmath,amsfonts,amsthm}
\usepackage[sfdefault]{roboto}
\usepackage[T1]{fontenc}
\usepackage{float}
\usepackage{multirow}
\usepackage{mathtools}
\usepackage{tikz}

\begin{document}
\maketitle

{
\setcounter{tocdepth}{2}
\tableofcontents
}
\section{Page 576: problem 2}\label{page-576-problem-2}

Consider a company that allows back ordering. That is, the company
notifies customers that a temporary stock-out exists and that their
order will be filled shortly. What considerations might argue for such a
policy? What effect does such a policy have on storage costs? Should
costs be assigned to stock-outs? Why? How would you make such an
assignment? What assumptions are implied by the model in Figure 13.7?
Suppose a ``loss of goodwill cost'' of w dollars per unit per day is
assigned to each stock-out. Compute the optimal order quantity Q* and
interpret your model.

\newpage

\section{Page 585: problem 2}\label{page-585-problem-2}

Find the local minimum value of the function

\(f(x,y) = 3x^2 + 6xy + 7y^2 - 2x + 4y\)

d/dx:

\(\frac{\partial}{\partial x}(3x^2 + 6xy + 7y^2 - 2x + 4y)\)\\
\(= 3(\frac{\partial}{\partial x}(x^2)) + 6y(\frac{\partial}{\partial x}(x)) + \frac{\partial}{\partial x}(7y^2) - 2 (\frac{\partial}{\partial x}(x)) + \frac{\partial}{\partial x}(4y)\)

Derivative of x is 1:

\(= 3(\frac{\partial}{\partial x}(x^2)) + 6y + \frac{\partial}{\partial x}(7y^2) - 2 + \frac{\partial}{\partial x}(4y)\)

Use the power rule, \(\frac{\partial}{\partial x}(x^n) = nx^{n-1}\),
where n = 2: \(\frac{\partial}{\partial x} (x^2)= 2x\) and derivative of
\(4y\) = 0:

\(= 3 (2x) + 6y + \frac{\partial}{\partial x}(7y^2) - 2\)

The derivative of \(7x^2\) = 0 :

\(\frac{\partial}{\partial x}(3x^2 + 6xy + 7y^2 - 2x + 47) =6x + 6y -2\)

d/dy:

\(\frac{\partial}{\partial y}(3x^2 + 6xy + 7y^2 - 2x + 47)\)\\
\(= (\frac{\partial}{\partial y}(3x^2)) + 6x(\frac{\partial}{\partial y}(y)) + 7\frac{\partial}{\partial y}(y^2) - (\frac{\partial}{\partial y}(-2x)) + 4\frac{\partial}{\partial y}(y)\)

The derivative of -2x, and \(3x^2\) = 0:

\(= 6x(\frac{\partial}{\partial y}(y)) + 7\frac{\partial}{\partial y}(y^2) + 4\frac{\partial}{\partial y}(y)\)

The derivative of y = 1:

\(= 6x + 7\frac{\partial}{\partial y}(y^2) + 4\)

Using the power rule :

\(0 = 6x + 7*2y + 4 = 6x + 14y + 4\)

Finding the local minima:\\
\(0 = 6x + 6y -2\)\\
\(-6y = 6x - 2\)\\
\(y = -x + \frac{1}{3}\)

\(0 = 6x + 14y + 4\)\\
\(-6x = 14y + 4\)\\
\(x = -\frac{14}{6}y - \frac{4}{6}\)\\
\(x = -\frac{14}{6}(-x + \frac{1}{3}) - \frac{4}{6}\)\\
\(x = \frac{13}{12}\)

\(y = - \frac{13}{12} + \frac{1}{3}\)

min at (x,y) = \((\frac{13}{12}, -\frac{1}{4})\)

\section{Page 591: problem 5}\label{page-591-problem-5}

\section{Page 599: problem 5}\label{page-599-problem-5}


\end{document}
