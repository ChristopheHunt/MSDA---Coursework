\documentclass[]{article}
\usepackage{lmodern}
\usepackage{amssymb,amsmath}
\usepackage{ifxetex,ifluatex}
\usepackage{fixltx2e} % provides \textsubscript
\ifnum 0\ifxetex 1\fi\ifluatex 1\fi=0 % if pdftex
  \usepackage[T1]{fontenc}
  \usepackage[utf8]{inputenc}
\else % if luatex or xelatex
  \ifxetex
    \usepackage{mathspec}
  \else
    \usepackage{fontspec}
  \fi
  \defaultfontfeatures{Ligatures=TeX,Scale=MatchLowercase}
\fi
% use upquote if available, for straight quotes in verbatim environments
\IfFileExists{upquote.sty}{\usepackage{upquote}}{}
% use microtype if available
\IfFileExists{microtype.sty}{%
\usepackage{microtype}
\UseMicrotypeSet[protrusion]{basicmath} % disable protrusion for tt fonts
}{}
\usepackage[margin=1in]{geometry}
\usepackage{hyperref}
\hypersetup{unicode=true,
            pdftitle={Homework 10},
            pdfauthor={Christophe Hunt},
            pdfborder={0 0 0},
            breaklinks=true}
\urlstyle{same}  % don't use monospace font for urls
\usepackage{graphicx,grffile}
\makeatletter
\def\maxwidth{\ifdim\Gin@nat@width>\linewidth\linewidth\else\Gin@nat@width\fi}
\def\maxheight{\ifdim\Gin@nat@height>\textheight\textheight\else\Gin@nat@height\fi}
\makeatother
% Scale images if necessary, so that they will not overflow the page
% margins by default, and it is still possible to overwrite the defaults
% using explicit options in \includegraphics[width, height, ...]{}
\setkeys{Gin}{width=\maxwidth,height=\maxheight,keepaspectratio}
\IfFileExists{parskip.sty}{%
\usepackage{parskip}
}{% else
\setlength{\parindent}{0pt}
\setlength{\parskip}{6pt plus 2pt minus 1pt}
}
\setlength{\emergencystretch}{3em}  % prevent overfull lines
\providecommand{\tightlist}{%
  \setlength{\itemsep}{0pt}\setlength{\parskip}{0pt}}
\setcounter{secnumdepth}{5}
% Redefines (sub)paragraphs to behave more like sections
\ifx\paragraph\undefined\else
\let\oldparagraph\paragraph
\renewcommand{\paragraph}[1]{\oldparagraph{#1}\mbox{}}
\fi
\ifx\subparagraph\undefined\else
\let\oldsubparagraph\subparagraph
\renewcommand{\subparagraph}[1]{\oldsubparagraph{#1}\mbox{}}
\fi

%%% Use protect on footnotes to avoid problems with footnotes in titles
\let\rmarkdownfootnote\footnote%
\def\footnote{\protect\rmarkdownfootnote}

%%% Change title format to be more compact
\usepackage{titling}

% Create subtitle command for use in maketitle
\newcommand{\subtitle}[1]{
  \posttitle{
    \begin{center}\large#1\end{center}
    }
}

\setlength{\droptitle}{-2em}
  \title{Homework 10}
  \pretitle{\vspace{\droptitle}\centering\huge}
  \posttitle{\par}
  \author{Christophe Hunt}
  \preauthor{\centering\large\emph}
  \postauthor{\par}
  \predate{\centering\large\emph}
  \postdate{\par}
  \date{April 3, 2017}

\usepackage{relsize}
\usepackage{setspace}
\usepackage{amsmath,amsfonts,amsthm}
\usepackage[sfdefault]{roboto}
\usepackage[T1]{fontenc}
\usepackage{float}
\usepackage{multirow}
\usepackage{mathtools}
\usepackage{tikz}

\begin{document}
\maketitle

{
\setcounter{tocdepth}{2}
\tableofcontents
}
\newpage

\section{Page 469: problem 3}\label{page-469-problem-3}

The following data were obtained for the growth of a sheep population
introduced into a new environment on the island of Tasmania (adapted
from J. Davidson ``On the Growth of the Sheep of Tasmania'' Trans. R.
Soc. S. Australia.)

\begin{table}[!h]
\centering
\caption{My caption}
\label{my-label}
\begin{tabular}{l|llllll}
t (year) & 1814 & 1824 & 1834 & 1844 & 1854 & 1864 \\ \hline
P (t) & 125 & 275 & 830 & 1200 & 1750 & 1650
\end{tabular}
\end{table}

\subsection{\texorpdfstring{a. Make an estimate of \(M\) by graphing
\(P(t)\).}{a. Make an estimate of M by graphing P(t).}}\label{a.-make-an-estimate-of-m-by-graphing-pt.}

\subsection{\texorpdfstring{b. Plot ln{[}P/(M-P){]} against \(t\). If a
logistic curve seems reasonable, estimate \(rM\) and
\(t\)*.}{b. Plot ln{[}P/(M-P){]} against t. If a logistic curve seems reasonable, estimate rM and t*.}}\label{b.-plot-lnpm-p-against-t.-if-a-logistic-curve-seems-reasonable-estimate-rm-and-t.}

\section{Page 478: problem 6}\label{page-478-problem-6}

Suggest other phenomena for which the model described in the text might
be used

\section{Page 481: problem 1}\label{page-481-problem-1}

\subsection{\texorpdfstring{a. Using the estimate that \(d_a\) =
0.054\(v^2\), where 0.054 has dimensions ft*hr\textsuperscript{2/Mi}2,
show that the constant \(k\) in Equation (11.29) has the value 19.9
ft/sec\^{}2.}{a. Using the estimate that d\_a = 0.054v\^{}2, where 0.054 has dimensions ft*hr2/Mi2, show that the constant k in Equation (11.29) has the value 19.9 ft/sec\^{}2.}}\label{a.-using-the-estimate-that-d_a-0.054v2-where-0.054-has-dimensions-fthr2mi2-show-that-the-constant-k-in-equation-11.29-has-the-value-19.9-ftsec2.}

\subsection{\texorpdfstring{b. Using the data in Table 4.4, plot \(d_b\)
in ft versuse v\^{}2/2 in ft\textsuperscript{2/sec}2 to estimate 1/\(k\)
directly.}{b. Using the data in Table 4.4, plot d\_b in ft versuse v\^{}2/2 in ft2/sec2 to estimate 1/k directly.}}\label{b.-using-the-data-in-table-4.4-plot-d_b-in-ft-versuse-v22-in-ft2sec2-to-estimate-1k-directly.}

\section{Page 522: problem 21}\label{page-522-problem-21}

Oxygen flows through on tube into a liter flasks filled with air, and
the mixture of oxygen and air (considered well stirred) escapes through
another tube. Assuming that air contains 21\% oxygen, what percentage of
oxygen will the flask contain after 5 L have passed through the intake
tube?

\section{Page 522: problem 22}\label{page-522-problem-22}

If the average person breathes 20 times per minute, exhaling each time
100 in\^{}3 of air containing 4\% carbon dioxide. Find the percentage of
carbon dioxide in the air of a 10,000 ft\^{}3 closed room 1 hr after a
class of 30 students enters. Assume that the air is fresh at the start,
that the ventilators admit 1000 ft\^{}3 of fresh air per minute, and
that the fresh air contains 0.04\% carbon dioxide.


\end{document}
