\documentclass[]{article}
\usepackage{lmodern}
\usepackage{amssymb,amsmath}
\usepackage{ifxetex,ifluatex}
\usepackage{fixltx2e} % provides \textsubscript
\ifnum 0\ifxetex 1\fi\ifluatex 1\fi=0 % if pdftex
  \usepackage[T1]{fontenc}
  \usepackage[utf8]{inputenc}
\else % if luatex or xelatex
  \ifxetex
    \usepackage{mathspec}
  \else
    \usepackage{fontspec}
  \fi
  \defaultfontfeatures{Ligatures=TeX,Scale=MatchLowercase}
\fi
% use upquote if available, for straight quotes in verbatim environments
\IfFileExists{upquote.sty}{\usepackage{upquote}}{}
% use microtype if available
\IfFileExists{microtype.sty}{%
\usepackage{microtype}
\UseMicrotypeSet[protrusion]{basicmath} % disable protrusion for tt fonts
}{}
\usepackage[margin=1in]{geometry}
\usepackage{hyperref}
\hypersetup{unicode=true,
            pdftitle={Homework 9},
            pdfauthor={Christophe Hunt},
            pdfborder={0 0 0},
            breaklinks=true}
\urlstyle{same}  % don't use monospace font for urls
\usepackage{graphicx,grffile}
\makeatletter
\def\maxwidth{\ifdim\Gin@nat@width>\linewidth\linewidth\else\Gin@nat@width\fi}
\def\maxheight{\ifdim\Gin@nat@height>\textheight\textheight\else\Gin@nat@height\fi}
\makeatother
% Scale images if necessary, so that they will not overflow the page
% margins by default, and it is still possible to overwrite the defaults
% using explicit options in \includegraphics[width, height, ...]{}
\setkeys{Gin}{width=\maxwidth,height=\maxheight,keepaspectratio}
\IfFileExists{parskip.sty}{%
\usepackage{parskip}
}{% else
\setlength{\parindent}{0pt}
\setlength{\parskip}{6pt plus 2pt minus 1pt}
}
\setlength{\emergencystretch}{3em}  % prevent overfull lines
\providecommand{\tightlist}{%
  \setlength{\itemsep}{0pt}\setlength{\parskip}{0pt}}
\setcounter{secnumdepth}{5}
% Redefines (sub)paragraphs to behave more like sections
\ifx\paragraph\undefined\else
\let\oldparagraph\paragraph
\renewcommand{\paragraph}[1]{\oldparagraph{#1}\mbox{}}
\fi
\ifx\subparagraph\undefined\else
\let\oldsubparagraph\subparagraph
\renewcommand{\subparagraph}[1]{\oldsubparagraph{#1}\mbox{}}
\fi

%%% Use protect on footnotes to avoid problems with footnotes in titles
\let\rmarkdownfootnote\footnote%
\def\footnote{\protect\rmarkdownfootnote}

%%% Change title format to be more compact
\usepackage{titling}

% Create subtitle command for use in maketitle
\newcommand{\subtitle}[1]{
  \posttitle{
    \begin{center}\large#1\end{center}
    }
}

\setlength{\droptitle}{-2em}
  \title{Homework 9}
  \pretitle{\vspace{\droptitle}\centering\huge}
  \posttitle{\par}
  \author{Christophe Hunt}
  \preauthor{\centering\large\emph}
  \postauthor{\par}
  \predate{\centering\large\emph}
  \postdate{\par}
  \date{April 1, 2017}

\usepackage{relsize}
\usepackage{setspace}
\usepackage{amsmath,amsfonts,amsthm}
\usepackage[sfdefault]{roboto}
\usepackage[T1]{fontenc}
\usepackage{float}
\usepackage{multirow}
\usepackage{mathtools}
\usepackage{tikz}

\begin{document}
\maketitle

{
\setcounter{tocdepth}{2}
\tableofcontents
}
\newpage

\section{Page 385: problem 1 a}\label{page-385-problem-1-a}

Using the definition provided for the movement diagram, determine
whether the following zero-sum games have a pure strategy Nash
equilibrium. If the game does have a pure strategy Nash equilibrium,
state the Nash equilibrium. Assume the row player is maximizing his
playoffs which are showing in the matrices below.

\begin{table}[!h]
\centering
\begin{tabular}{lllc}
 &  & \multicolumn{2}{l}{Colin} \\ \cline{3-4}
 &  & C1 & \multicolumn{1}{l}{C2} \\ \hline
Rose & R1 & \multicolumn{1}{c}{10} & 10 \\
 & R2 & \multicolumn{1}{c}{5} & 0 \\ \hline
\end{tabular}
\end{table}

\section{Page 385: problem 1 c}\label{page-385-problem-1-c}

Using the definition provided for the movement diagram, determine
whether the following zero-sum games have a pure strategy Nash
equilibrium. If the game does have a pure strategy Nash equilibrium,
state the Nash equilibrium. Assume the row player is maximizing his
playoffs which are showing in the matrices below.

\begin{table}[!h]
\centering
\begin{tabular}{lllc}
 &  & \multicolumn{2}{l}{Colin} \\ \cline{3-4}
 &  & C1 & \multicolumn{1}{l}{C2} \\ \hline
Rose & R1 & \multicolumn{1}{c}{1/2} & 1/2 \\
 & R2 & \multicolumn{1}{c}{1} & 0 \\ \hline
\end{tabular}
\end{table}

\section{Page 404: problem 2 a}\label{page-404-problem-2-a}

For problems a-g build a linear programming model for each player's
decisions and solve it both geometrically and algebraically. Assume the
row player is maximizing his playoffs which are showing in the matrices
below.

\begin{table}[!h]
\centering
\begin{tabular}{lllc}
 &  & \multicolumn{2}{l}{Colin} \\ \cline{3-4}
 &  & C1 & \multicolumn{1}{l}{C2} \\ \hline
Rose & R1 & \multicolumn{1}{c}{10} & 10 \\
 & R2 & \multicolumn{1}{c}{5} & 0 \\ \hline
\end{tabular}
\end{table}

\section{Page 413: problem 3}\label{page-413-problem-3}

We are considering three alternatives A, B, and C under states of nature
1, 2 ,3, and 4, set up and solve both the investor's and nature's game:

\begin{table}[!h]
\centering
\begin{tabular}{crrrr}
\multicolumn{5}{l}{\textbf{States of Nature}} \\ \hline
\begin{tabular}[c]{@{}c@{}}Investor's choices \\ Alternatives\end{tabular} & \multicolumn{1}{c}{Condition \#1} & \multicolumn{1}{c}{Condition \#2} & \multicolumn{1}{c}{Condition \#3} & \multicolumn{1}{l}{Condition \#4} \\ \hline
A & 1100 & 900 & 400 & 300 \\
B & 850 & 1500 & 1000 & 500 \\
C & 700 & 1200 & 500 & 900 \\ \hline
\end{tabular}
\end{table}

\section{Page 420: problem 1}\label{page-420-problem-1}

In the following porlbems, use the maximim and minimax method and
movement diagram to determine if any pure strategy solution exist.
Assume the row player is maximizing his payoffs which are shown in the
matrices below.

\begin{table}[!h]
\centering
\begin{tabular}{lllc}
 &  & \multicolumn{2}{l}{Colin} \\ \cline{3-4}
 &  & C1 & \multicolumn{1}{l}{C2} \\ \hline
Rose & R1 & \multicolumn{1}{c}{10} & 10 \\
 & R2 & \multicolumn{1}{c}{5} & 0 \\ \hline
\end{tabular}
\end{table}

\section{Page 428: problem 3}\label{page-428-problem-3}

Using the alternative methods (a) equating expected value and (b) mthods
of oddments to find the solutions to the following games. Assume the row
player is maximizing his payoffs which are shown in the matrices below.

\begin{table}[!h]
\centering
\begin{tabular}{lllc}
 &  & \multicolumn{2}{l}{Colin} \\ \cline{3-4}
 &  & C1 & \multicolumn{1}{l}{C2} \\ \hline
Rose & R1 & \multicolumn{1}{c}{0.5} & 0.3 \\
 & R2 & \multicolumn{1}{c}{0.6} & 1 \\ \hline
\end{tabular}
\end{table}

\section{Page 440: problem 2}\label{page-440-problem-2}

Use movements diagrams to find all the stable outcomes in Problems 1
through 5. Then use strategic moves (using Table 10.2) to determine if
Rose can get a better outcome.

\begin{table}[!h]
\centering
\begin{tabular}{lllc}
 &  & \multicolumn{2}{l}{Colin} \\ \cline{3-4}
 &  & C1 & \multicolumn{1}{l}{C2} \\ \hline
Rose & R1 & \multicolumn{1}{c}{(1,2)} & (3,1) \\
 & R2 & \multicolumn{1}{c}{(2, 4)} & (4, 3)\\ \hline
\end{tabular}
\end{table}

\section{Page 454: problem 3}\label{page-454-problem-3}


\end{document}
