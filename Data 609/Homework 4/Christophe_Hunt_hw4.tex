\documentclass[]{article}
\usepackage{lmodern}
\usepackage{amssymb,amsmath}
\usepackage{ifxetex,ifluatex}
\usepackage{fixltx2e} % provides \textsubscript
\ifnum 0\ifxetex 1\fi\ifluatex 1\fi=0 % if pdftex
  \usepackage[T1]{fontenc}
  \usepackage[utf8]{inputenc}
\else % if luatex or xelatex
  \ifxetex
    \usepackage{mathspec}
  \else
    \usepackage{fontspec}
  \fi
  \defaultfontfeatures{Ligatures=TeX,Scale=MatchLowercase}
\fi
% use upquote if available, for straight quotes in verbatim environments
\IfFileExists{upquote.sty}{\usepackage{upquote}}{}
% use microtype if available
\IfFileExists{microtype.sty}{%
\usepackage{microtype}
\UseMicrotypeSet[protrusion]{basicmath} % disable protrusion for tt fonts
}{}
\usepackage[margin=1in]{geometry}
\usepackage{hyperref}
\hypersetup{unicode=true,
            pdftitle={Homework 4},
            pdfauthor={Christophe Hunt},
            pdfborder={0 0 0},
            breaklinks=true}
\urlstyle{same}  % don't use monospace font for urls
\usepackage{color}
\usepackage{fancyvrb}
\newcommand{\VerbBar}{|}
\newcommand{\VERB}{\Verb[commandchars=\\\{\}]}
\DefineVerbatimEnvironment{Highlighting}{Verbatim}{commandchars=\\\{\}}
% Add ',fontsize=\small' for more characters per line
\usepackage{framed}
\definecolor{shadecolor}{RGB}{248,248,248}
\newenvironment{Shaded}{\begin{snugshade}}{\end{snugshade}}
\newcommand{\KeywordTok}[1]{\textcolor[rgb]{0.13,0.29,0.53}{\textbf{{#1}}}}
\newcommand{\DataTypeTok}[1]{\textcolor[rgb]{0.13,0.29,0.53}{{#1}}}
\newcommand{\DecValTok}[1]{\textcolor[rgb]{0.00,0.00,0.81}{{#1}}}
\newcommand{\BaseNTok}[1]{\textcolor[rgb]{0.00,0.00,0.81}{{#1}}}
\newcommand{\FloatTok}[1]{\textcolor[rgb]{0.00,0.00,0.81}{{#1}}}
\newcommand{\ConstantTok}[1]{\textcolor[rgb]{0.00,0.00,0.00}{{#1}}}
\newcommand{\CharTok}[1]{\textcolor[rgb]{0.31,0.60,0.02}{{#1}}}
\newcommand{\SpecialCharTok}[1]{\textcolor[rgb]{0.00,0.00,0.00}{{#1}}}
\newcommand{\StringTok}[1]{\textcolor[rgb]{0.31,0.60,0.02}{{#1}}}
\newcommand{\VerbatimStringTok}[1]{\textcolor[rgb]{0.31,0.60,0.02}{{#1}}}
\newcommand{\SpecialStringTok}[1]{\textcolor[rgb]{0.31,0.60,0.02}{{#1}}}
\newcommand{\ImportTok}[1]{{#1}}
\newcommand{\CommentTok}[1]{\textcolor[rgb]{0.56,0.35,0.01}{\textit{{#1}}}}
\newcommand{\DocumentationTok}[1]{\textcolor[rgb]{0.56,0.35,0.01}{\textbf{\textit{{#1}}}}}
\newcommand{\AnnotationTok}[1]{\textcolor[rgb]{0.56,0.35,0.01}{\textbf{\textit{{#1}}}}}
\newcommand{\CommentVarTok}[1]{\textcolor[rgb]{0.56,0.35,0.01}{\textbf{\textit{{#1}}}}}
\newcommand{\OtherTok}[1]{\textcolor[rgb]{0.56,0.35,0.01}{{#1}}}
\newcommand{\FunctionTok}[1]{\textcolor[rgb]{0.00,0.00,0.00}{{#1}}}
\newcommand{\VariableTok}[1]{\textcolor[rgb]{0.00,0.00,0.00}{{#1}}}
\newcommand{\ControlFlowTok}[1]{\textcolor[rgb]{0.13,0.29,0.53}{\textbf{{#1}}}}
\newcommand{\OperatorTok}[1]{\textcolor[rgb]{0.81,0.36,0.00}{\textbf{{#1}}}}
\newcommand{\BuiltInTok}[1]{{#1}}
\newcommand{\ExtensionTok}[1]{{#1}}
\newcommand{\PreprocessorTok}[1]{\textcolor[rgb]{0.56,0.35,0.01}{\textit{{#1}}}}
\newcommand{\AttributeTok}[1]{\textcolor[rgb]{0.77,0.63,0.00}{{#1}}}
\newcommand{\RegionMarkerTok}[1]{{#1}}
\newcommand{\InformationTok}[1]{\textcolor[rgb]{0.56,0.35,0.01}{\textbf{\textit{{#1}}}}}
\newcommand{\WarningTok}[1]{\textcolor[rgb]{0.56,0.35,0.01}{\textbf{\textit{{#1}}}}}
\newcommand{\AlertTok}[1]{\textcolor[rgb]{0.94,0.16,0.16}{{#1}}}
\newcommand{\ErrorTok}[1]{\textcolor[rgb]{0.64,0.00,0.00}{\textbf{{#1}}}}
\newcommand{\NormalTok}[1]{{#1}}
\usepackage{longtable,booktabs}
\usepackage{graphicx,grffile}
\makeatletter
\def\maxwidth{\ifdim\Gin@nat@width>\linewidth\linewidth\else\Gin@nat@width\fi}
\def\maxheight{\ifdim\Gin@nat@height>\textheight\textheight\else\Gin@nat@height\fi}
\makeatother
% Scale images if necessary, so that they will not overflow the page
% margins by default, and it is still possible to overwrite the defaults
% using explicit options in \includegraphics[width, height, ...]{}
\setkeys{Gin}{width=\maxwidth,height=\maxheight,keepaspectratio}
\IfFileExists{parskip.sty}{%
\usepackage{parskip}
}{% else
\setlength{\parindent}{0pt}
\setlength{\parskip}{6pt plus 2pt minus 1pt}
}
\setlength{\emergencystretch}{3em}  % prevent overfull lines
\providecommand{\tightlist}{%
  \setlength{\itemsep}{0pt}\setlength{\parskip}{0pt}}
\setcounter{secnumdepth}{5}
% Redefines (sub)paragraphs to behave more like sections
\ifx\paragraph\undefined\else
\let\oldparagraph\paragraph
\renewcommand{\paragraph}[1]{\oldparagraph{#1}\mbox{}}
\fi
\ifx\subparagraph\undefined\else
\let\oldsubparagraph\subparagraph
\renewcommand{\subparagraph}[1]{\oldsubparagraph{#1}\mbox{}}
\fi

%%% Use protect on footnotes to avoid problems with footnotes in titles
\let\rmarkdownfootnote\footnote%
\def\footnote{\protect\rmarkdownfootnote}

%%% Change title format to be more compact
\usepackage{titling}

% Create subtitle command for use in maketitle
\newcommand{\subtitle}[1]{
  \posttitle{
    \begin{center}\large#1\end{center}
    }
}

\setlength{\droptitle}{-2em}
  \title{Homework 4}
  \pretitle{\vspace{\droptitle}\centering\huge}
  \posttitle{\par}
  \author{Christophe Hunt}
  \preauthor{\centering\large\emph}
  \postauthor{\par}
  \predate{\centering\large\emph}
  \postdate{\par}
  \date{February 20, 2017}

\usepackage{relsize}
\usepackage{setspace}
\usepackage{amsmath,amsfonts,amsthm}
\usepackage[sfdefault]{roboto}
\usepackage[T1]{fontenc}
\usepackage{float}

\begin{document}
\maketitle

{
\setcounter{tocdepth}{2}
\tableofcontents
}
\newpage

\section{Page 191: problem 3}\label{page-191-problem-3}

Using Monte Carlo Simulation, write an algorithm to calculate an
approximation to \(\pi\) by considering the number of random points
selected inside the quarter circle
\[Q:x^2 +y^2 = 1, x \geq 0, y \geq 0\] where the quarter circle is taken
to be inside the square \[S : 0 \leq x \leq 1~and~0 \leq y \leq 1\] Use
the equation \(\frac{\pi}{4}=area\); \(\frac{Q}{area}S\).

\begin{Shaded}
\begin{Highlighting}[]
\KeywordTok{set.seed}\NormalTok{(}\DecValTok{1234}\NormalTok{)}

\NormalTok{monte_carlo <-}\StringTok{ }\NormalTok{function(n)\{}
  \NormalTok{counter =}\StringTok{ }\DecValTok{0}
  \NormalTok{for (i in }\DecValTok{1}\NormalTok{:n)\{}
    \NormalTok{y <-}\StringTok{ }\KeywordTok{runif}\NormalTok{(}\DecValTok{1}\NormalTok{, }\DecValTok{0}\NormalTok{, }\DecValTok{1}\NormalTok{)}
    \NormalTok{x <-}\StringTok{ }\KeywordTok{runif}\NormalTok{(}\DecValTok{1}\NormalTok{, }\DecValTok{0}\NormalTok{, }\DecValTok{1}\NormalTok{)}
    \NormalTok{if ((x^}\DecValTok{2} \NormalTok{+}\StringTok{ }\NormalTok{y^}\DecValTok{2}\NormalTok{) <}\StringTok{ }\DecValTok{1}\NormalTok{)\{}
      \NormalTok{counter <-}\StringTok{ }\NormalTok{counter +}\StringTok{ }\DecValTok{1} 
    \NormalTok{\} else \{}
      \NormalTok{counter <-}\StringTok{ }\NormalTok{counter}
    \NormalTok{\}}
  \NormalTok{\}}
  \KeywordTok{return}\NormalTok{(counter)}
\NormalTok{\}}
\end{Highlighting}
\end{Shaded}

\begin{Shaded}
\begin{Highlighting}[]
\NormalTok{n <-}\StringTok{ }\DecValTok{500}
\NormalTok{(}\KeywordTok{monte_carlo}\NormalTok{(n)/n)*}\DecValTok{4}
\end{Highlighting}
\end{Shaded}

\begin{verbatim}
## [1] 3.104
\end{verbatim}

\begin{Shaded}
\begin{Highlighting}[]
\NormalTok{n <-}\StringTok{ }\DecValTok{50000}
\NormalTok{(}\KeywordTok{monte_carlo}\NormalTok{(n)/n)*}\DecValTok{4}
\end{Highlighting}
\end{Shaded}

\begin{verbatim}
## [1] 3.14312
\end{verbatim}

The results of the algorithm get us very close to the value of \(\pi\)
which is 3.1415927 as calculated in R.

\newpage

\section{Page 194: problem 1}\label{page-194-problem-1}

Use the middle-square method to generate.

\begin{Shaded}
\begin{Highlighting}[]
\NormalTok{middle_square <-}\StringTok{ }\NormalTok{function(n, seed) \{}
          \KeywordTok{suppressMessages}\NormalTok{(}\KeywordTok{require}\NormalTok{(stringr))}
          \NormalTok{results <-}\StringTok{ }\KeywordTok{list}\NormalTok{(}\StringTok{"instances"} \NormalTok{=}\StringTok{ }\NormalTok{n, }\StringTok{"starting_seed"} \NormalTok{=}\StringTok{ }\NormalTok{seed)}
          \NormalTok{for (i in }\DecValTok{1}\NormalTok{:n)\{}
            \NormalTok{j <-}\StringTok{ }\NormalTok{(seed^}\DecValTok{2}\NormalTok{)}
              \NormalTok{if (}\KeywordTok{nchar}\NormalTok{(j) <}\StringTok{ }\DecValTok{8}\NormalTok{)\{}
                  \NormalTok{j <-}\StringTok{ }\KeywordTok{str_pad}\NormalTok{(j, }\DecValTok{8}\NormalTok{, }\DataTypeTok{pad =} \StringTok{"0"}\NormalTok{)}
              \NormalTok{\} else if (}\KeywordTok{nchar}\NormalTok{(j) >}\StringTok{ }\DecValTok{8}\NormalTok{) \{}
                \NormalTok{j <-}\StringTok{ }\KeywordTok{substr}\NormalTok{(j, }\DecValTok{1}\NormalTok{, }\DecValTok{8}\NormalTok{)}
              \NormalTok{\}}
            \NormalTok{t <-}\StringTok{ }\NormalTok{j }
            \NormalTok{j <-}\StringTok{ }\KeywordTok{substr}\NormalTok{(j, }\DecValTok{3}\NormalTok{, }\DecValTok{6}\NormalTok{)}
            \NormalTok{seed <-}\StringTok{ }\KeywordTok{as.numeric}\NormalTok{(j)}
            \NormalTok{results[[}\KeywordTok{length}\NormalTok{(results)+}\DecValTok{1}\NormalTok{]] <-}\StringTok{ }\KeywordTok{list}\NormalTok{(}\StringTok{"number"} \NormalTok{=}\StringTok{ }\NormalTok{t, }\StringTok{"new_seed"} \NormalTok{=}\StringTok{ }\NormalTok{seed)}
          \NormalTok{\}}
         \KeywordTok{return}\NormalTok{(results)}
\NormalTok{\}}
\end{Highlighting}
\end{Shaded}

\subsection{\texorpdfstring{a. 10 random numbers using
\(x_0 = 1009\)}{a. 10 random numbers using x\_0 = 1009}}\label{a.-10-random-numbers-using-x_0-1009}

\begin{Shaded}
\begin{Highlighting}[]
\KeywordTok{middle_square}\NormalTok{(}\DecValTok{10}\NormalTok{, }\DecValTok{1009}\NormalTok{)}
\end{Highlighting}
\end{Shaded}

\$instances {[}1{]} 10

\$starting\_seed {[}1{]} 1009

{[}{[}3{]}{]}{[}{[}3{]}{]}\$number {[}1{]} ``01018081''

{[}{[}3{]}{]}\$new\_seed {[}1{]} 180

{[}{[}4{]}{]}{[}{[}4{]}{]}\$number {[}1{]} ``00032400''

{[}{[}4{]}{]}\$new\_seed {[}1{]} 324

{[}{[}5{]}{]}{[}{[}5{]}{]}\$number {[}1{]} ``00104976''

{[}{[}5{]}{]}\$new\_seed {[}1{]} 1049

{[}{[}6{]}{]}{[}{[}6{]}{]}\$number {[}1{]} ``01100401''

{[}{[}6{]}{]}\$new\_seed {[}1{]} 1004

{[}{[}7{]}{]}{[}{[}7{]}{]}\$number {[}1{]} ``01008016''

{[}{[}7{]}{]}\$new\_seed {[}1{]} 80

{[}{[}8{]}{]}{[}{[}8{]}{]}\$number {[}1{]} ``00006400''

{[}{[}8{]}{]}\$new\_seed {[}1{]} 64

{[}{[}9{]}{]}{[}{[}9{]}{]}\$number {[}1{]} ``00004096''

{[}{[}9{]}{]}\$new\_seed {[}1{]} 40

{[}{[}10{]}{]}{[}{[}10{]}{]}\$number {[}1{]} ``00001600''

{[}{[}10{]}{]}\$new\_seed {[}1{]} 16

{[}{[}11{]}{]}{[}{[}11{]}{]}\$number {[}1{]} ``00000256''

{[}{[}11{]}{]}\$new\_seed {[}1{]} 2

{[}{[}12{]}{]}{[}{[}12{]}{]}\$number {[}1{]} ``00000004''

{[}{[}12{]}{]}\$new\_seed {[}1{]} 0

\subsection{\texorpdfstring{b. 20 random numbers using
\(x_0 = 653217\)}{b. 20 random numbers using x\_0 = 653217}}\label{b.-20-random-numbers-using-x_0-653217}

\begin{Shaded}
\begin{Highlighting}[]
\KeywordTok{middle_square}\NormalTok{(}\DecValTok{20}\NormalTok{, }\DecValTok{653217}\NormalTok{)}
\end{Highlighting}
\end{Shaded}

\$instances {[}1{]} 20

\$starting\_seed {[}1{]} 653217

{[}{[}3{]}{]}{[}{[}3{]}{]}\$number {[}1{]} ``42669244''

{[}{[}3{]}{]}\$new\_seed {[}1{]} 6692

{[}{[}4{]}{]}{[}{[}4{]}{]}\$number {[}1{]} 44782864

{[}{[}4{]}{]}\$new\_seed {[}1{]} 7828

{[}{[}5{]}{]}{[}{[}5{]}{]}\$number {[}1{]} 61277584

{[}{[}5{]}{]}\$new\_seed {[}1{]} 2775

{[}{[}6{]}{]}{[}{[}6{]}{]}\$number {[}1{]} ``07700625''

{[}{[}6{]}{]}\$new\_seed {[}1{]} 7006

{[}{[}7{]}{]}{[}{[}7{]}{]}\$number {[}1{]} 49084036

{[}{[}7{]}{]}\$new\_seed {[}1{]} 840

{[}{[}8{]}{]}{[}{[}8{]}{]}\$number {[}1{]} ``00705600''

{[}{[}8{]}{]}\$new\_seed {[}1{]} 7056

{[}{[}9{]}{]}{[}{[}9{]}{]}\$number {[}1{]} 49787136

{[}{[}9{]}{]}\$new\_seed {[}1{]} 7871

{[}{[}10{]}{]}{[}{[}10{]}{]}\$number {[}1{]} 61952641

{[}{[}10{]}{]}\$new\_seed {[}1{]} 9526

{[}{[}11{]}{]}{[}{[}11{]}{]}\$number {[}1{]} 90744676

{[}{[}11{]}{]}\$new\_seed {[}1{]} 7446

{[}{[}12{]}{]}{[}{[}12{]}{]}\$number {[}1{]} 55442916

{[}{[}12{]}{]}\$new\_seed {[}1{]} 4429

{[}{[}13{]}{]}{[}{[}13{]}{]}\$number {[}1{]} 19616041

{[}{[}13{]}{]}\$new\_seed {[}1{]} 6160

{[}{[}14{]}{]}{[}{[}14{]}{]}\$number {[}1{]} 37945600

{[}{[}14{]}{]}\$new\_seed {[}1{]} 9456

{[}{[}15{]}{]}{[}{[}15{]}{]}\$number {[}1{]} 89415936

{[}{[}15{]}{]}\$new\_seed {[}1{]} 4159

{[}{[}16{]}{]}{[}{[}16{]}{]}\$number {[}1{]} 17297281

{[}{[}16{]}{]}\$new\_seed {[}1{]} 2972

{[}{[}17{]}{]}{[}{[}17{]}{]}\$number {[}1{]} ``08832784''

{[}{[}17{]}{]}\$new\_seed {[}1{]} 8327

{[}{[}18{]}{]}{[}{[}18{]}{]}\$number {[}1{]} 69338929

{[}{[}18{]}{]}\$new\_seed {[}1{]} 3389

{[}{[}19{]}{]}{[}{[}19{]}{]}\$number {[}1{]} 11485321

{[}{[}19{]}{]}\$new\_seed {[}1{]} 4853

{[}{[}20{]}{]}{[}{[}20{]}{]}\$number {[}1{]} 23551609

{[}{[}20{]}{]}\$new\_seed {[}1{]} 5516

{[}{[}21{]}{]}{[}{[}21{]}{]}\$number {[}1{]} 30426256

{[}{[}21{]}{]}\$new\_seed {[}1{]} 4262

{[}{[}22{]}{]}{[}{[}22{]}{]}\$number {[}1{]} 18164644

{[}{[}22{]}{]}\$new\_seed {[}1{]} 1646

\subsection{\texorpdfstring{c. 15 random numbers using
\(x_0 = 3043\)}{c. 15 random numbers using x\_0 = 3043}}\label{c.-15-random-numbers-using-x_0-3043}

\begin{Shaded}
\begin{Highlighting}[]
\KeywordTok{middle_square}\NormalTok{(}\DecValTok{15}\NormalTok{, }\DecValTok{3043}\NormalTok{)}
\end{Highlighting}
\end{Shaded}

\$instances {[}1{]} 15

\$starting\_seed {[}1{]} 3043

{[}{[}3{]}{]}{[}{[}3{]}{]}\$number {[}1{]} ``09259849''

{[}{[}3{]}{]}\$new\_seed {[}1{]} 2598

{[}{[}4{]}{]}{[}{[}4{]}{]}\$number {[}1{]} ``06749604''

{[}{[}4{]}{]}\$new\_seed {[}1{]} 7496

{[}{[}5{]}{]}{[}{[}5{]}{]}\$number {[}1{]} 56190016

{[}{[}5{]}{]}\$new\_seed {[}1{]} 1900

{[}{[}6{]}{]}{[}{[}6{]}{]}\$number {[}1{]} ``03610000''

{[}{[}6{]}{]}\$new\_seed {[}1{]} 6100

{[}{[}7{]}{]}{[}{[}7{]}{]}\$number {[}1{]} 37210000

{[}{[}7{]}{]}\$new\_seed {[}1{]} 2100

{[}{[}8{]}{]}{[}{[}8{]}{]}\$number {[}1{]} ``04410000''

{[}{[}8{]}{]}\$new\_seed {[}1{]} 4100

{[}{[}9{]}{]}{[}{[}9{]}{]}\$number {[}1{]} 16810000

{[}{[}9{]}{]}\$new\_seed {[}1{]} 8100

{[}{[}10{]}{]}{[}{[}10{]}{]}\$number {[}1{]} 65610000

{[}{[}10{]}{]}\$new\_seed {[}1{]} 6100

{[}{[}11{]}{]}{[}{[}11{]}{]}\$number {[}1{]} 37210000

{[}{[}11{]}{]}\$new\_seed {[}1{]} 2100

{[}{[}12{]}{]}{[}{[}12{]}{]}\$number {[}1{]} ``04410000''

{[}{[}12{]}{]}\$new\_seed {[}1{]} 4100

{[}{[}13{]}{]}{[}{[}13{]}{]}\$number {[}1{]} 16810000

{[}{[}13{]}{]}\$new\_seed {[}1{]} 8100

{[}{[}14{]}{]}{[}{[}14{]}{]}\$number {[}1{]} 65610000

{[}{[}14{]}{]}\$new\_seed {[}1{]} 6100

{[}{[}15{]}{]}{[}{[}15{]}{]}\$number {[}1{]} 37210000

{[}{[}15{]}{]}\$new\_seed {[}1{]} 2100

{[}{[}16{]}{]}{[}{[}16{]}{]}\$number {[}1{]} ``04410000''

{[}{[}16{]}{]}\$new\_seed {[}1{]} 4100

{[}{[}17{]}{]}{[}{[}17{]}{]}\$number {[}1{]} 16810000

{[}{[}17{]}{]}\$new\_seed {[}1{]} 8100

\subsection{d. Comment about the results of each sequence. Was there
cycling? Did each sequence degenerate
rapidly?}\label{d.-comment-about-the-results-of-each-sequence.-was-there-cycling-did-each-sequence-degenerate-rapidly}

The first sequence degenerated very rapidly. It reached 0 by the 10
iteration and would not be suitable for use in a very sensitive
analysis. It is concerning that the algorithm can reach 0 so quickly.
Also, in c. we see the values begin repeating and this again is very
concerning for a sensitive analysis. It may be difficult to spot this
type of cycling in a simulation and may provide inaccurate results.
However, in b. we do see the power of the randomness generation in the
algorithm. As the literature says the random generation by this method
is no longer a reliable algorithm.

\newpage

\section{Page 199: problem 4}\label{page-199-problem-4}

Given loaded dice according to the following distribution, use Monte
Carlo simulation to simulate the sum of 300 rolls of two unfair dice.

\begin{Shaded}
\begin{Highlighting}[]
\KeywordTok{library}\NormalTok{(knitr)}
\NormalTok{Roll <-}\StringTok{ }\KeywordTok{c}\NormalTok{(}\DecValTok{1}\NormalTok{:}\DecValTok{6}\NormalTok{)}
\NormalTok{Die_1 <-}\StringTok{ }\KeywordTok{c}\NormalTok{(.}\DecValTok{1}\NormalTok{,.}\DecValTok{1}\NormalTok{,.}\DecValTok{2}\NormalTok{,.}\DecValTok{3}\NormalTok{,.}\DecValTok{2}\NormalTok{,.}\DecValTok{1}\NormalTok{)}
\NormalTok{Die_2 <-}\StringTok{ }\KeywordTok{c}\NormalTok{(.}\DecValTok{3}\NormalTok{,.}\DecValTok{1}\NormalTok{,.}\DecValTok{2}\NormalTok{,.}\DecValTok{1}\NormalTok{,.}\DecValTok{05}\NormalTok{,.}\DecValTok{25}\NormalTok{)}
\KeywordTok{kable}\NormalTok{(}\KeywordTok{as.data.frame}\NormalTok{(}\KeywordTok{cbind}\NormalTok{(Roll, Die_1, Die_2)))}
\end{Highlighting}
\end{Shaded}

\begin{longtable}[]{@{}rrr@{}}
\toprule
Roll & Die\_1 & Die\_2\tabularnewline
\midrule
\endhead
1 & 0.1 & 0.30\tabularnewline
2 & 0.1 & 0.10\tabularnewline
3 & 0.2 & 0.20\tabularnewline
4 & 0.3 & 0.10\tabularnewline
5 & 0.2 & 0.05\tabularnewline
6 & 0.1 & 0.25\tabularnewline
\bottomrule
\end{longtable}

\begin{Shaded}
\begin{Highlighting}[]
\NormalTok{unfair_roll <-}\StringTok{ }\NormalTok{function(n, probs)\{}
    \NormalTok{if (}\KeywordTok{length}\NormalTok{(probs) !=}\StringTok{ }\DecValTok{6}\NormalTok{)\{}
      \KeywordTok{print}\NormalTok{(}\StringTok{"missing probabilities for all 6 sides"}\NormalTok{)}
      \NormalTok{break}
    \NormalTok{\}}
    \NormalTok{results <-}\StringTok{ }\KeywordTok{list}\NormalTok{(}\StringTok{"1"} \NormalTok{=}\StringTok{ }\DecValTok{0}\NormalTok{, }\StringTok{"2"} \NormalTok{=}\StringTok{ }\DecValTok{0} \NormalTok{, }\StringTok{"3"} \NormalTok{=}\StringTok{ }\DecValTok{0}\NormalTok{, }\StringTok{"4"} \NormalTok{=}\StringTok{ }\DecValTok{0}\NormalTok{, }\StringTok{"5"} \NormalTok{=}\StringTok{ }\DecValTok{0}\NormalTok{, }\StringTok{"6"} \NormalTok{=}\StringTok{ }\DecValTok{0}\NormalTok{)}
         \NormalTok{for (i in }\DecValTok{1}\NormalTok{:n)\{}
       \NormalTok{j <-}\StringTok{ }\DecValTok{1} 
       \NormalTok{x <-}\StringTok{ }\KeywordTok{runif}\NormalTok{(j, }\DecValTok{0}\NormalTok{, j)}
       \NormalTok{if (x ==}\StringTok{ }\DecValTok{0}\NormalTok{)\{}
        \NormalTok{results[[j]] <-}\StringTok{ }\NormalTok{results[[j]] +}\StringTok{ }\DecValTok{1}
       \NormalTok{\} else if (x >}\StringTok{ }\DecValTok{0} \NormalTok{&}\StringTok{ }\NormalTok{x <=}\StringTok{ }\NormalTok{probs[[j]])\{}
        \NormalTok{results[[j]] <-}\StringTok{ }\NormalTok{results[[j]] +}\StringTok{ }\DecValTok{1}
        \NormalTok{next}
       \NormalTok{\} else \{}
       \NormalTok{j <-}\StringTok{ }\NormalTok{j +}\StringTok{ }\DecValTok{1}
       \NormalTok{while ((x >}\StringTok{ }\KeywordTok{sum}\NormalTok{(probs[}\DecValTok{0}\NormalTok{:(j}\DecValTok{-1}\NormalTok{)]) &}\StringTok{ }\NormalTok{x <=}\StringTok{ }\KeywordTok{sum}\NormalTok{(probs[}\DecValTok{0}\NormalTok{:j])) !=}\StringTok{ }\OtherTok{TRUE}\NormalTok{) \{}
       \NormalTok{j <-}\StringTok{ }\NormalTok{j +}\StringTok{ }\DecValTok{1}
       \NormalTok{\}}
       \NormalTok{results[[j]] <-}\StringTok{ }\NormalTok{results[[j]] +}\StringTok{ }\DecValTok{1}
       \NormalTok{\}}
     \NormalTok{\}}
    \NormalTok{for (i in }\DecValTok{1}\NormalTok{:}\DecValTok{6}\NormalTok{)\{}
     \KeywordTok{names}\NormalTok{(results[[i]]) <-}\StringTok{ "occurences"}
     \NormalTok{results[[i]][[}\StringTok{"probability"}\NormalTok{]] <-}\StringTok{ }\KeywordTok{round}\NormalTok{(results[[i]]/n,}\DecValTok{2}\NormalTok{)}
    \NormalTok{\}}
  \KeywordTok{return}\NormalTok{(results)}
\NormalTok{\}}
\end{Highlighting}
\end{Shaded}

\begin{Shaded}
\begin{Highlighting}[]
\KeywordTok{unfair_roll}\NormalTok{(}\DecValTok{300}\NormalTok{, Die_1)}
\end{Highlighting}
\end{Shaded}

\begin{verbatim}
## $`1`
##  occurences probability 
##       27.00        0.09 
## 
## $`2`
##  occurences probability 
##       34.00        0.11 
## 
## $`3`
##  occurences probability 
##       56.00        0.19 
## 
## $`4`
##  occurences probability 
##        90.0         0.3 
## 
## $`5`
##  occurences probability 
##        59.0         0.2 
## 
## $`6`
##  occurences probability 
##       34.00        0.11
\end{verbatim}

\begin{Shaded}
\begin{Highlighting}[]
\KeywordTok{unfair_roll}\NormalTok{(}\DecValTok{300}\NormalTok{, Die_2)}
\end{Highlighting}
\end{Shaded}

\begin{verbatim}
## $`1`
##  occurences probability 
##       82.00        0.27 
## 
## $`2`
##  occurences probability 
##       24.00        0.08 
## 
## $`3`
##  occurences probability 
##       69.00        0.23 
## 
## $`4`
##  occurences probability 
##       27.00        0.09 
## 
## $`5`
##  occurences probability 
##       12.00        0.04 
## 
## $`6`
##  occurences probability 
##       86.00        0.29
\end{verbatim}

The simulation provides rolls of the dice as we would expect. The
probability is very similar to the distribution provided and is a very
useful technique to generate a simulation of random rolls of a dice.

\section{Page 211: problem 3}\label{page-211-problem-3}

In many situations, the time \(T\) between deliveries and the order
quantity \(Q\) is not fixed. Instead, an order is placed for a specific
amount of gasoline. Depending on how many orders are placed in a given
time interval, the time to fill an order varies. You have no reason to
believe that the performance of the delivery operation will change.
Therefore, you have examined records for the past 100 deliveries and
found the following lag times, or extra days, required to fill your
order:

\begin{Shaded}
\begin{Highlighting}[]
\NormalTok{Lag_time <-}\StringTok{ }\KeywordTok{c}\NormalTok{(}\DecValTok{2}\NormalTok{:}\DecValTok{7}\NormalTok{)}
\NormalTok{number_of_occurrences <-}\StringTok{ }\KeywordTok{c}\NormalTok{(}\DecValTok{10}\NormalTok{, }\DecValTok{25}\NormalTok{, }\DecValTok{30}\NormalTok{, }\DecValTok{20}\NormalTok{, }\DecValTok{13}\NormalTok{, }\DecValTok{2}\NormalTok{)}
\NormalTok{X <-}\StringTok{ }\KeywordTok{as.data.frame}\NormalTok{(}\KeywordTok{cbind}\NormalTok{(Lag_time, number_of_occurrences))}
\KeywordTok{kable}\NormalTok{(}\KeywordTok{rbind}\NormalTok{(X, }\KeywordTok{c}\NormalTok{(}\StringTok{" "}\NormalTok{, }\KeywordTok{sum}\NormalTok{(number_of_occurrences))))}
\end{Highlighting}
\end{Shaded}

\begin{longtable}[]{@{}ll@{}}
\toprule
Lag\_time & number\_of\_occurrences\tabularnewline
\midrule
\endhead
2 & 10\tabularnewline
3 & 25\tabularnewline
4 & 30\tabularnewline
5 & 20\tabularnewline
6 & 13\tabularnewline
7 & 2\tabularnewline
& 100\tabularnewline
\bottomrule
\end{longtable}

Construct a Monte Carlo simulation for the lag time sub model. If you
have a handheld calculator or computer available, test your sub model by
running 1000 trials and comparing the number of occurrences of the
various lag times with the historical data.

\begin{Shaded}
\begin{Highlighting}[]
\NormalTok{Lag_time <-}\StringTok{ }\KeywordTok{c}\NormalTok{(}\DecValTok{2}\NormalTok{:}\DecValTok{7}\NormalTok{)}
\NormalTok{number_of_occurrences <-}\StringTok{ }\KeywordTok{c}\NormalTok{(}\DecValTok{10}\NormalTok{, }\DecValTok{25}\NormalTok{, }\DecValTok{30}\NormalTok{, }\DecValTok{20}\NormalTok{, }\DecValTok{13}\NormalTok{, }\DecValTok{2}\NormalTok{)}

\NormalTok{lag_time <-}\StringTok{ }\NormalTok{function(n, probs, outcomes)\{}
        \NormalTok{results <-}\StringTok{ }\KeywordTok{list}\NormalTok{()}
        \NormalTok{for (i in }\DecValTok{1}\NormalTok{:}\KeywordTok{length}\NormalTok{(outcomes))\{}
         \NormalTok{results[[i]] <-}\StringTok{ }\DecValTok{0}
         \KeywordTok{names}\NormalTok{(results[[i]]) <-}\StringTok{ }\NormalTok{(}\KeywordTok{paste}\NormalTok{(}\StringTok{"# of deliveries on day"}\NormalTok{, outcomes[[i]]))}
        \NormalTok{\}}
     \NormalTok{for (i in }\DecValTok{1}\NormalTok{:n)\{}
       \NormalTok{j <-}\StringTok{ }\DecValTok{1} 
       \NormalTok{x <-}\StringTok{ }\KeywordTok{runif}\NormalTok{(j, }\DecValTok{0}\NormalTok{, j)}
       \NormalTok{if (x ==}\StringTok{ }\DecValTok{0}\NormalTok{)\{}
        \NormalTok{results[[j]] <-}\StringTok{ }\NormalTok{results[[j]] +}\StringTok{ }\DecValTok{1}
       \NormalTok{\} else if (x >}\StringTok{ }\DecValTok{0} \NormalTok{&}\StringTok{ }\NormalTok{x <=}\StringTok{ }\NormalTok{probs[[j]])\{}
        \NormalTok{results[[j]] <-}\StringTok{ }\NormalTok{results[[j]] +}\StringTok{ }\DecValTok{1}
        \NormalTok{next}
       \NormalTok{\} else \{}
       \NormalTok{j <-}\StringTok{ }\NormalTok{j +}\StringTok{ }\DecValTok{1}
       \NormalTok{while ((x >}\StringTok{ }\KeywordTok{sum}\NormalTok{(probs[}\DecValTok{0}\NormalTok{:(j}\DecValTok{-1}\NormalTok{)]) &}\StringTok{ }\NormalTok{x <=}\StringTok{ }\KeywordTok{sum}\NormalTok{(probs[}\DecValTok{0}\NormalTok{:j])) !=}\StringTok{ }\OtherTok{TRUE}\NormalTok{) \{}
       \NormalTok{j <-}\StringTok{ }\NormalTok{j +}\StringTok{ }\DecValTok{1}
       \NormalTok{\}}
       \NormalTok{results[[j]] <-}\StringTok{ }\NormalTok{results[[j]] +}\StringTok{ }\DecValTok{1}
       \NormalTok{\}}
     \NormalTok{\}}
    \NormalTok{for (i in }\DecValTok{1}\NormalTok{:}\KeywordTok{length}\NormalTok{(results))\{}
     \NormalTok{results[[i]][[}\StringTok{"probability"}\NormalTok{]] <-}\StringTok{ }\KeywordTok{round}\NormalTok{(results[[i]]/n,}\DecValTok{2}\NormalTok{)}
    \NormalTok{\}}
  \KeywordTok{return}\NormalTok{(results)}
\NormalTok{\}}
\NormalTok{probs <-}\StringTok{ }\NormalTok{(number_of_occurrences/}\KeywordTok{sum}\NormalTok{(number_of_occurrences))}
\KeywordTok{lag_time}\NormalTok{(}\DecValTok{1000}\NormalTok{, }\DataTypeTok{probs =} \NormalTok{probs , }\DataTypeTok{outcomes =} \NormalTok{Lag_time)}
\end{Highlighting}
\end{Shaded}

\begin{verbatim}
## [[1]]
## # of deliveries on day 2              probability 
##                    102.0                      0.1 
## 
## [[2]]
## # of deliveries on day 3              probability 
##                   237.00                     0.24 
## 
## [[3]]
## # of deliveries on day 4              probability 
##                    299.0                      0.3 
## 
## [[4]]
## # of deliveries on day 5              probability 
##                   211.00                     0.21 
## 
## [[5]]
## # of deliveries on day 6              probability 
##                   125.00                     0.12 
## 
## [[6]]
## # of deliveries on day 7              probability 
##                    26.00                     0.03
\end{verbatim}

The results follow the distribution provided, it is easy to compare in
this situation because the total observations is 100 making the
probabilities quick to spot. It is a useful tool to visualize how many
deliveries over the course of 1000 may fall into the lag range and we
can plan accordingly.


\end{document}
