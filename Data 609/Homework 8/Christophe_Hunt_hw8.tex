\documentclass[]{article}
\usepackage{lmodern}
\usepackage{amssymb,amsmath}
\usepackage{ifxetex,ifluatex}
\usepackage{fixltx2e} % provides \textsubscript
\ifnum 0\ifxetex 1\fi\ifluatex 1\fi=0 % if pdftex
  \usepackage[T1]{fontenc}
  \usepackage[utf8]{inputenc}
\else % if luatex or xelatex
  \ifxetex
    \usepackage{mathspec}
  \else
    \usepackage{fontspec}
  \fi
  \defaultfontfeatures{Ligatures=TeX,Scale=MatchLowercase}
\fi
% use upquote if available, for straight quotes in verbatim environments
\IfFileExists{upquote.sty}{\usepackage{upquote}}{}
% use microtype if available
\IfFileExists{microtype.sty}{%
\usepackage{microtype}
\UseMicrotypeSet[protrusion]{basicmath} % disable protrusion for tt fonts
}{}
\usepackage[margin=1in]{geometry}
\usepackage{hyperref}
\hypersetup{unicode=true,
            pdftitle={Homework 8},
            pdfauthor={Christophe Hunt},
            pdfborder={0 0 0},
            breaklinks=true}
\urlstyle{same}  % don't use monospace font for urls
\usepackage{graphicx,grffile}
\makeatletter
\def\maxwidth{\ifdim\Gin@nat@width>\linewidth\linewidth\else\Gin@nat@width\fi}
\def\maxheight{\ifdim\Gin@nat@height>\textheight\textheight\else\Gin@nat@height\fi}
\makeatother
% Scale images if necessary, so that they will not overflow the page
% margins by default, and it is still possible to overwrite the defaults
% using explicit options in \includegraphics[width, height, ...]{}
\setkeys{Gin}{width=\maxwidth,height=\maxheight,keepaspectratio}
\IfFileExists{parskip.sty}{%
\usepackage{parskip}
}{% else
\setlength{\parindent}{0pt}
\setlength{\parskip}{6pt plus 2pt minus 1pt}
}
\setlength{\emergencystretch}{3em}  % prevent overfull lines
\providecommand{\tightlist}{%
  \setlength{\itemsep}{0pt}\setlength{\parskip}{0pt}}
\setcounter{secnumdepth}{5}
% Redefines (sub)paragraphs to behave more like sections
\ifx\paragraph\undefined\else
\let\oldparagraph\paragraph
\renewcommand{\paragraph}[1]{\oldparagraph{#1}\mbox{}}
\fi
\ifx\subparagraph\undefined\else
\let\oldsubparagraph\subparagraph
\renewcommand{\subparagraph}[1]{\oldsubparagraph{#1}\mbox{}}
\fi

%%% Use protect on footnotes to avoid problems with footnotes in titles
\let\rmarkdownfootnote\footnote%
\def\footnote{\protect\rmarkdownfootnote}

%%% Change title format to be more compact
\usepackage{titling}

% Create subtitle command for use in maketitle
\newcommand{\subtitle}[1]{
  \posttitle{
    \begin{center}\large#1\end{center}
    }
}

\setlength{\droptitle}{-2em}
  \title{Homework 8}
  \pretitle{\vspace{\droptitle}\centering\huge}
  \posttitle{\par}
  \author{Christophe Hunt}
  \preauthor{\centering\large\emph}
  \postauthor{\par}
  \predate{\centering\large\emph}
  \postdate{\par}
  \date{March 21, 2017}

\usepackage{relsize}
\usepackage{setspace}
\usepackage{amsmath,amsfonts,amsthm}
\usepackage[sfdefault]{roboto}
\usepackage[T1]{fontenc}
\usepackage{float}
\usepackage{multirow}
\usepackage{mathtools}
\usepackage{tikz,forest}
\usetikzlibrary{arrows.meta}
\forestset{
    .style={
        for tree={
            base=bottom,
            child anchor=north,
            align=center,
            s sep+=1cm,
    straight edge/.style={
        edge path={\noexpand\path[\forestoption{edge},thick,-{Latex}]
        (!u.parent anchor) -- (.child anchor);}
    },
    if n children={0}
        {tier=word, draw, thick, rectangle}
        {draw, diamond, thick, aspect=2},
    if n=1{%
        edge path={\noexpand\path[\forestoption{edge},thick,-{Latex}]
        (!u.parent anchor) -| (.child anchor) node[pos=.2, above] {Y};}
        }{
        edge path={\noexpand\path[\forestoption{edge},thick,-{Latex}]
        (!u.parent anchor) -| (.child anchor) node[pos=.2, above] {N};}
        }
        }
    }
}

\begin{document}
\maketitle

{
\setcounter{tocdepth}{2}
\tableofcontents
}
\newpage

\section{Page 347: problem 4}\label{page-347-problem-4}

We have engaged in a business venture. Assume the probability of success
is \(P(s) = \frac{2}{5}\); further assume that if we are successful we
make \$55,000, and we are unsuccessful we lose \$1,750. Find the
expected value of then business venture.

\begin{quote}
\(E = (\$55,000)*0.4 + (-\$1,750)*0.6\) = \(\$20,950\)
\end{quote}

\section{Page 347: problem 6}\label{page-347-problem-6}

Consider a firm handling concessions for a sporting vent. The firm's
manager needs to know whether to stock up with coffee or cola and is
formulation policies for specific weather predictions. A local agreement
restricts the firm to selling only one type of beverage. The firm
estimates a \$1,500 profit selling cola if the weather is cold and a
\$5,000 profit selling cola if the weather is warm. The firm also
estimates a \$4,000 profit selling coffee if it is cold and a \$1000
profit selling coffee if the weather is warm. The weather forecast says
that there is a 30\% of a cold front; otherwise, the weather will be
warm. Build a decision tree to assist with the decision. What should the
firm handling concessions do?

\begin{center}
\begin{forest} 
[$Weather$, tikz={\draw[{Latex}-, thick] (.north) --++ (0,1);}
    [$Warm$,edge label={node[midway,left] {.70\,\,\,}} 
        [$Cola$
            [\$5000] 
        ]    
        [$Coffee$
            [\$1000] 
        ]    
    ]   
    [$Cold$,edge label={node[midway,right] {\,\,\,.30}} 
        [$Cola$
            [\$1500] 
        ]    
        [$Coffee$
            [\$4000] 
        ] 
    ]   
] 
\end{forest}
\end{center}

\begin{quote}
Therefore the expected values are as follows:\\
\(E(Cola) = .7 * 5000 + .3 * 1500 = \$3,900\)\\
\(E(Coffee) = .7 * 1000 + .3 * 4000 = \$1,900\)
\end{quote}

\begin{quote}
The logical conclusion is to sell Cola since it has over twice the
expected value.
\end{quote}

\newpage

\section{Page 355: problem 3}\label{page-355-problem-3}

The financial success of a ski resort in Squaw valley is dependent on
the amount of early snowfall in the fall and winter months. If the
snowfall is greater than 40 inches, the resort always has a successful
ski season. If the snow is between 30 and 40 inches, the resort has a
moderate season, and if the snowfall is less than 30 inches, the season
is poor, and the resort will lose money. The seasonal snow probabilities
from the weather service are displayed in the following table with the
expected revenue for the previous 10 seasons. A hotel chain has offered
to lease the resort during the winter for \$100,000. You must decide
whether to operate yourself or lease the resort. Build a decision tree
to assist in the decision.

\begin{center}
\begin{forest} 
[$Resort$, tikz={\draw[{Latex}-, thick] (.north) --++ (0,1);}
    [$Lease$,edge label={node[midway,left] {1\,\,\,}} 
            [\$100000] 
    ]   
    [$Keep$ 
        [$Sucessful$,edge label={node[midway,left] {.40\,\,\,}}
            [\$280000] 
        ]    
        [$Moderate$,edge label={node[midway,right] {\,\,.20}}
            [\$100000] 
        ] 
        [$Poor$,edge label={node[midway,right] {\,\,\,.40}}
            [-\$40000] 
        ] 
    ]   
] 
\end{forest}
\end{center}

\begin{quote}
\(E(Lease) = 1 * 100000 = \$100,000\)\\
\(E(Keep) = .4 * 280000 + .2 * 100000 + .4 * -40000 = \$116,000\)
\end{quote}

\begin{quote}
Since our expected value is greater if we keep the property, then the
ski resort should keep the property.
\end{quote}

\newpage

\section{Page 364: problem 3}\label{page-364-problem-3}

A big private oil company must decide whether to drill in the Gulf of
Mexico. It costs \$1 million to drill, and if oil is found its value is
estimated at \$6 million. At present, the oil company believes that
there is a 45\% chance that oil is present. Before drilling begins, the
big private oil company can hire a geologist for \$100,000 to obtain
samples and test for oil. There is only about a 60\% chance that the
geologist will issue a favorable report. Given that the geologist does
issue a favorable report, there is an 85\% chance that there is oil.
Given an unfavorable report, there is a 22\% chance that there is oil.
Determine what the big private oil company should do.

\begin{center}
\begin{forest} 
[$Drill$, tikz={\draw[{Latex}-, thick] (.north) --++ (0,1);}
    [$hire~geologist$ 
        [$favorable~report$,edge label={node[midway,left] {\,\,\,.60}}
        [$oil$,edge label={node[midway,left] {.85\,\,\,}}
            [\$4900000] 
        ] 
        [$no~oil$,edge label={node[midway,right] {\,\,\,.15}}
            [-\$110000] 
        ] 
        ] 
        [$unfavorable~report$,edge label={node[midway,right] {.40\,\,\,}}
        [$oil$,edge label={node[midway,left] {.22\,\,\,}}
            [\$4900000] 
        ] 
        [$no~oil$,edge label={node[midway,right] {\,\,\,.78}}
            [-\$1100000] 
        ] 
        ] 
    ]   
    [$no~geologist$ 
        [$oil$,edge label={node[midway,left] {.45\,\,\,}}
            [\$5000000] 
        ] 
        [$no~oil$,edge label={node[midway,right] {\,\,\,.55}}
            [-\$1000000] 
        ] 
    ]   
] 
\end{forest}
\end{center}

\small{E(hire geologist) = (.6 * .85 * 4900000) + (.6 * .15 * -1100000) + (.4 * .22 * 4900000) + (.4 * .78 * -1100000) = \$2,488,000}

\small{E(no geologist) = (.45 * 5000000) + (.55 * -1000000) = \$1,700,000}

\begin{quote}
Since the expected value is greatest when we hire a geologist (at
\$2,488,000), the private big oil company should hire a geologist.
\end{quote}


\end{document}
