\documentclass[]{article}
\usepackage{lmodern}
\usepackage{amssymb,amsmath}
\usepackage{ifxetex,ifluatex}
\usepackage{fixltx2e} % provides \textsubscript
\ifnum 0\ifxetex 1\fi\ifluatex 1\fi=0 % if pdftex
  \usepackage[T1]{fontenc}
  \usepackage[utf8]{inputenc}
\else % if luatex or xelatex
  \ifxetex
    \usepackage{mathspec}
  \else
    \usepackage{fontspec}
  \fi
  \defaultfontfeatures{Ligatures=TeX,Scale=MatchLowercase}
\fi
% use upquote if available, for straight quotes in verbatim environments
\IfFileExists{upquote.sty}{\usepackage{upquote}}{}
% use microtype if available
\IfFileExists{microtype.sty}{%
\usepackage{microtype}
\UseMicrotypeSet[protrusion]{basicmath} % disable protrusion for tt fonts
}{}
\usepackage[margin=1in]{geometry}
\usepackage{hyperref}
\hypersetup{unicode=true,
            pdftitle={Homework 12},
            pdfauthor={Christophe Hunt},
            pdfborder={0 0 0},
            breaklinks=true}
\urlstyle{same}  % don't use monospace font for urls
\usepackage{graphicx,grffile}
\makeatletter
\def\maxwidth{\ifdim\Gin@nat@width>\linewidth\linewidth\else\Gin@nat@width\fi}
\def\maxheight{\ifdim\Gin@nat@height>\textheight\textheight\else\Gin@nat@height\fi}
\makeatother
% Scale images if necessary, so that they will not overflow the page
% margins by default, and it is still possible to overwrite the defaults
% using explicit options in \includegraphics[width, height, ...]{}
\setkeys{Gin}{width=\maxwidth,height=\maxheight,keepaspectratio}
\IfFileExists{parskip.sty}{%
\usepackage{parskip}
}{% else
\setlength{\parindent}{0pt}
\setlength{\parskip}{6pt plus 2pt minus 1pt}
}
\setlength{\emergencystretch}{3em}  % prevent overfull lines
\providecommand{\tightlist}{%
  \setlength{\itemsep}{0pt}\setlength{\parskip}{0pt}}
\setcounter{secnumdepth}{5}
% Redefines (sub)paragraphs to behave more like sections
\ifx\paragraph\undefined\else
\let\oldparagraph\paragraph
\renewcommand{\paragraph}[1]{\oldparagraph{#1}\mbox{}}
\fi
\ifx\subparagraph\undefined\else
\let\oldsubparagraph\subparagraph
\renewcommand{\subparagraph}[1]{\oldsubparagraph{#1}\mbox{}}
\fi

%%% Use protect on footnotes to avoid problems with footnotes in titles
\let\rmarkdownfootnote\footnote%
\def\footnote{\protect\rmarkdownfootnote}

%%% Change title format to be more compact
\usepackage{titling}

% Create subtitle command for use in maketitle
\newcommand{\subtitle}[1]{
  \posttitle{
    \begin{center}\large#1\end{center}
    }
}

\setlength{\droptitle}{-2em}
  \title{Homework 12}
  \pretitle{\vspace{\droptitle}\centering\huge}
  \posttitle{\par}
  \author{Christophe Hunt}
  \preauthor{\centering\large\emph}
  \postauthor{\par}
  \predate{\centering\large\emph}
  \postdate{\par}
  \date{April 22, 2017}

\usepackage{relsize}
\usepackage{setspace}
\usepackage{amsmath,amsfonts,amsthm}
\usepackage[sfdefault]{roboto}
\usepackage[T1]{fontenc}
\usepackage{float}
\usepackage{multirow}
\usepackage{mathtools}
\usepackage{tikz}

\begin{document}
\maketitle

{
\setcounter{tocdepth}{2}
\tableofcontents
}
\newpage

\section{Page 529: problem 1}\label{page-529-problem-1}

Verify that the given function pair is a solution to the first-order
system.

\(x = -e^t\), \(y = e^t\)\\
\(\frac{dx}{dt} = -y\), \(\frac{dy}{dt} = -x\)

\(\frac{dx}{dt} = \frac{d}{dt}(-e^t) = e^t = y\) ;
\(\frac{dx}{dt} = -y\)

\(\frac{dy}{dt} = \frac{d}{dt}(e^t) = -e^t = x\) ;
\(\frac{dy}{dt} = -x\)

\newpage

\section{Page 529: problem 6}\label{page-529-problem-6}

Find and classify the rest points of the given autonomous system.

\(\frac{dx}{dt} = -(y-1)\), \(\frac{dy}{dt} = x-2\)

The rest point of the system is a point in the phase plane for which
\(f(x,y) = 0 and g(x,y) = 0\), then both the derivatives
\(\frac{dx}{dt} = 0\) and \(\frac{dy}{dt}= 0\).

when \(y = 1\), \(\frac{dx}{dt} = -(1-1)\); \(\frac{dx}{dt}= 0\)\\
when \(x = 2\), \(\frac{dy}{dt} = 2 - 2\); \(\frac{dy}{dt} = 0\)

\((2,1)\) is the rest point of the autonomous system
\(\frac{dx}{dt} = -(y-1)\), \(\frac{dy}{dt} = x-2\)

\section{Page 546: problem 1}\label{page-546-problem-1}

Apply the first and second derivative tests to the function

\(f(y) = y^a/e^{by}\) to show that \(f(y) = y^a/e^{by}\)

\section{Page 566: problem 1}\label{page-566-problem-1}

Use Euler's method to solve the first-order system subject to the
specificed intial conditions. Use the given step size \(\Delta{t}\) and
calculate the first three approximateions
\((x_1, y_1), (x_2, y_2), and (x_3, y_3)\). Then repeat your
calculations for \(\Delta t/2\). Compare your approximations with the
values of the given analytic solutions


\end{document}
