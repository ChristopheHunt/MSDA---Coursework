\documentclass[]{article}
\usepackage{lmodern}
\usepackage{amssymb,amsmath}
\usepackage{ifxetex,ifluatex}
\usepackage{fixltx2e} % provides \textsubscript
\ifnum 0\ifxetex 1\fi\ifluatex 1\fi=0 % if pdftex
  \usepackage[T1]{fontenc}
  \usepackage[utf8]{inputenc}
\else % if luatex or xelatex
  \ifxetex
    \usepackage{mathspec}
  \else
    \usepackage{fontspec}
  \fi
  \defaultfontfeatures{Ligatures=TeX,Scale=MatchLowercase}
\fi
% use upquote if available, for straight quotes in verbatim environments
\IfFileExists{upquote.sty}{\usepackage{upquote}}{}
% use microtype if available
\IfFileExists{microtype.sty}{%
\usepackage{microtype}
\UseMicrotypeSet[protrusion]{basicmath} % disable protrusion for tt fonts
}{}
\usepackage[margin=1in]{geometry}
\usepackage{hyperref}
\hypersetup{unicode=true,
            pdftitle={Homework 6},
            pdfauthor={Christophe Hunt},
            pdfborder={0 0 0},
            breaklinks=true}
\urlstyle{same}  % don't use monospace font for urls
\usepackage{graphicx,grffile}
\makeatletter
\def\maxwidth{\ifdim\Gin@nat@width>\linewidth\linewidth\else\Gin@nat@width\fi}
\def\maxheight{\ifdim\Gin@nat@height>\textheight\textheight\else\Gin@nat@height\fi}
\makeatother
% Scale images if necessary, so that they will not overflow the page
% margins by default, and it is still possible to overwrite the defaults
% using explicit options in \includegraphics[width, height, ...]{}
\setkeys{Gin}{width=\maxwidth,height=\maxheight,keepaspectratio}
\IfFileExists{parskip.sty}{%
\usepackage{parskip}
}{% else
\setlength{\parindent}{0pt}
\setlength{\parskip}{6pt plus 2pt minus 1pt}
}
\setlength{\emergencystretch}{3em}  % prevent overfull lines
\providecommand{\tightlist}{%
  \setlength{\itemsep}{0pt}\setlength{\parskip}{0pt}}
\setcounter{secnumdepth}{5}
% Redefines (sub)paragraphs to behave more like sections
\ifx\paragraph\undefined\else
\let\oldparagraph\paragraph
\renewcommand{\paragraph}[1]{\oldparagraph{#1}\mbox{}}
\fi
\ifx\subparagraph\undefined\else
\let\oldsubparagraph\subparagraph
\renewcommand{\subparagraph}[1]{\oldsubparagraph{#1}\mbox{}}
\fi

%%% Use protect on footnotes to avoid problems with footnotes in titles
\let\rmarkdownfootnote\footnote%
\def\footnote{\protect\rmarkdownfootnote}

%%% Change title format to be more compact
\usepackage{titling}

% Create subtitle command for use in maketitle
\newcommand{\subtitle}[1]{
  \posttitle{
    \begin{center}\large#1\end{center}
    }
}

\setlength{\droptitle}{-2em}
  \title{Homework 6}
  \pretitle{\vspace{\droptitle}\centering\huge}
  \posttitle{\par}
  \author{Christophe Hunt}
  \preauthor{\centering\large\emph}
  \postauthor{\par}
  \predate{\centering\large\emph}
  \postdate{\par}
  \date{March 12, 2017}

\usepackage{relsize}
\usepackage{setspace}
\usepackage{amsmath,amsfonts,amsthm}
\usepackage[sfdefault]{roboto}
\usepackage[T1]{fontenc}
\usepackage{float}
\usepackage{multirow}

\begin{document}
\maketitle

{
\setcounter{tocdepth}{2}
\tableofcontents
}
\section{Problem Set 1}\label{problem-set-1}

\subsection{(1) When you roll a fair die 3 times, how many possible
outcomes are
there?}\label{when-you-roll-a-fair-die-3-times-how-many-possible-outcomes-are-there}

\subsection{(2) What is the probability of getting a sum total of 3 when
you roll a die two
times?}\label{what-is-the-probability-of-getting-a-sum-total-of-3-when-you-roll-a-die-two-times}

\subsection{(3) Assume a room of 25 strangers. What is the probability
that two of them have the same birthday? Assume that all birthdays are
equally likely and equal to 1/365 each. What happens to this probability
when there are 50 people in the
room?}\label{assume-a-room-of-25-strangers.-what-is-the-probability-that-two-of-them-have-the-same-birthday-assume-that-all-birthdays-are-equally-likely-and-equal-to-1365-each.-what-happens-to-this-probability-when-there-are-50-people-in-the-room}

\section{Problem Set 2}\label{problem-set-2}

Sometimes you cannot compute the probability of an outcome by measuring
the sample space and examining the symmetries of the underlying physical
phenomenon, as you could do when you rolled die or picked a card from a
shu???ed deck. You have to estimate probabilities by other means. For
instance, when you have to compute the probability of various english
words, it is not possible to do it by examination of the sample space as
it is too large. You have to resort to empirical techniques to get a
good enough estimate. One such approach would be to take a large corpus
of documents and from those documents, count the number of occurrences
of a particular character or word and then base your estimate on that.

Write a program to take a document in English and print out the
estimated probabilities for each of the words that occur in that
document. Your program should take in a ???le containing a large
document and write out the probabilities of each of the words that
appear in that document. Please remove all punctuation (quotes, commas,
hyphens etc) and convert the words to lower case before you perform your
calculations.

Extend your program to calculate the probability of two words occurring
adjacent to each other. It should take in a document, and two words (say
the and for) and compute the probability of each of the words occurring
in the document and the joint probability of both of them occurring
together. The order of the two words is not important.

Use the accompanying document for your testing purposes. Compare your
probabilities of various words with the Time Magazine corpus:
\url{http://corpus.byu.edu/time/}


\end{document}
