\documentclass[]{article}
\usepackage{lmodern}
\usepackage{amssymb,amsmath}
\usepackage{ifxetex,ifluatex}
\usepackage{fixltx2e} % provides \textsubscript
\ifnum 0\ifxetex 1\fi\ifluatex 1\fi=0 % if pdftex
  \usepackage[T1]{fontenc}
  \usepackage[utf8]{inputenc}
\else % if luatex or xelatex
  \ifxetex
    \usepackage{mathspec}
  \else
    \usepackage{fontspec}
  \fi
  \defaultfontfeatures{Ligatures=TeX,Scale=MatchLowercase}
\fi
% use upquote if available, for straight quotes in verbatim environments
\IfFileExists{upquote.sty}{\usepackage{upquote}}{}
% use microtype if available
\IfFileExists{microtype.sty}{%
\usepackage{microtype}
\UseMicrotypeSet[protrusion]{basicmath} % disable protrusion for tt fonts
}{}
\usepackage[margin=1in]{geometry}
\usepackage{hyperref}
\hypersetup{unicode=true,
            pdftitle={Homework 13},
            pdfauthor={Christophe Hunt},
            pdfborder={0 0 0},
            breaklinks=true}
\urlstyle{same}  % don't use monospace font for urls
\usepackage{color}
\usepackage{fancyvrb}
\newcommand{\VerbBar}{|}
\newcommand{\VERB}{\Verb[commandchars=\\\{\}]}
\DefineVerbatimEnvironment{Highlighting}{Verbatim}{commandchars=\\\{\}}
% Add ',fontsize=\small' for more characters per line
\usepackage{framed}
\definecolor{shadecolor}{RGB}{248,248,248}
\newenvironment{Shaded}{\begin{snugshade}}{\end{snugshade}}
\newcommand{\KeywordTok}[1]{\textcolor[rgb]{0.13,0.29,0.53}{\textbf{{#1}}}}
\newcommand{\DataTypeTok}[1]{\textcolor[rgb]{0.13,0.29,0.53}{{#1}}}
\newcommand{\DecValTok}[1]{\textcolor[rgb]{0.00,0.00,0.81}{{#1}}}
\newcommand{\BaseNTok}[1]{\textcolor[rgb]{0.00,0.00,0.81}{{#1}}}
\newcommand{\FloatTok}[1]{\textcolor[rgb]{0.00,0.00,0.81}{{#1}}}
\newcommand{\ConstantTok}[1]{\textcolor[rgb]{0.00,0.00,0.00}{{#1}}}
\newcommand{\CharTok}[1]{\textcolor[rgb]{0.31,0.60,0.02}{{#1}}}
\newcommand{\SpecialCharTok}[1]{\textcolor[rgb]{0.00,0.00,0.00}{{#1}}}
\newcommand{\StringTok}[1]{\textcolor[rgb]{0.31,0.60,0.02}{{#1}}}
\newcommand{\VerbatimStringTok}[1]{\textcolor[rgb]{0.31,0.60,0.02}{{#1}}}
\newcommand{\SpecialStringTok}[1]{\textcolor[rgb]{0.31,0.60,0.02}{{#1}}}
\newcommand{\ImportTok}[1]{{#1}}
\newcommand{\CommentTok}[1]{\textcolor[rgb]{0.56,0.35,0.01}{\textit{{#1}}}}
\newcommand{\DocumentationTok}[1]{\textcolor[rgb]{0.56,0.35,0.01}{\textbf{\textit{{#1}}}}}
\newcommand{\AnnotationTok}[1]{\textcolor[rgb]{0.56,0.35,0.01}{\textbf{\textit{{#1}}}}}
\newcommand{\CommentVarTok}[1]{\textcolor[rgb]{0.56,0.35,0.01}{\textbf{\textit{{#1}}}}}
\newcommand{\OtherTok}[1]{\textcolor[rgb]{0.56,0.35,0.01}{{#1}}}
\newcommand{\FunctionTok}[1]{\textcolor[rgb]{0.00,0.00,0.00}{{#1}}}
\newcommand{\VariableTok}[1]{\textcolor[rgb]{0.00,0.00,0.00}{{#1}}}
\newcommand{\ControlFlowTok}[1]{\textcolor[rgb]{0.13,0.29,0.53}{\textbf{{#1}}}}
\newcommand{\OperatorTok}[1]{\textcolor[rgb]{0.81,0.36,0.00}{\textbf{{#1}}}}
\newcommand{\BuiltInTok}[1]{{#1}}
\newcommand{\ExtensionTok}[1]{{#1}}
\newcommand{\PreprocessorTok}[1]{\textcolor[rgb]{0.56,0.35,0.01}{\textit{{#1}}}}
\newcommand{\AttributeTok}[1]{\textcolor[rgb]{0.77,0.63,0.00}{{#1}}}
\newcommand{\RegionMarkerTok}[1]{{#1}}
\newcommand{\InformationTok}[1]{\textcolor[rgb]{0.56,0.35,0.01}{\textbf{\textit{{#1}}}}}
\newcommand{\WarningTok}[1]{\textcolor[rgb]{0.56,0.35,0.01}{\textbf{\textit{{#1}}}}}
\newcommand{\AlertTok}[1]{\textcolor[rgb]{0.94,0.16,0.16}{{#1}}}
\newcommand{\ErrorTok}[1]{\textcolor[rgb]{0.64,0.00,0.00}{\textbf{{#1}}}}
\newcommand{\NormalTok}[1]{{#1}}
\usepackage{graphicx,grffile}
\makeatletter
\def\maxwidth{\ifdim\Gin@nat@width>\linewidth\linewidth\else\Gin@nat@width\fi}
\def\maxheight{\ifdim\Gin@nat@height>\textheight\textheight\else\Gin@nat@height\fi}
\makeatother
% Scale images if necessary, so that they will not overflow the page
% margins by default, and it is still possible to overwrite the defaults
% using explicit options in \includegraphics[width, height, ...]{}
\setkeys{Gin}{width=\maxwidth,height=\maxheight,keepaspectratio}
\IfFileExists{parskip.sty}{%
\usepackage{parskip}
}{% else
\setlength{\parindent}{0pt}
\setlength{\parskip}{6pt plus 2pt minus 1pt}
}
\setlength{\emergencystretch}{3em}  % prevent overfull lines
\providecommand{\tightlist}{%
  \setlength{\itemsep}{0pt}\setlength{\parskip}{0pt}}
\setcounter{secnumdepth}{5}
% Redefines (sub)paragraphs to behave more like sections
\ifx\paragraph\undefined\else
\let\oldparagraph\paragraph
\renewcommand{\paragraph}[1]{\oldparagraph{#1}\mbox{}}
\fi
\ifx\subparagraph\undefined\else
\let\oldsubparagraph\subparagraph
\renewcommand{\subparagraph}[1]{\oldsubparagraph{#1}\mbox{}}
\fi

%%% Use protect on footnotes to avoid problems with footnotes in titles
\let\rmarkdownfootnote\footnote%
\def\footnote{\protect\rmarkdownfootnote}

%%% Change title format to be more compact
\usepackage{titling}

% Create subtitle command for use in maketitle
\newcommand{\subtitle}[1]{
  \posttitle{
    \begin{center}\large#1\end{center}
    }
}

\setlength{\droptitle}{-2em}
  \title{Homework 13}
  \pretitle{\vspace{\droptitle}\centering\huge}
  \posttitle{\par}
  \author{Christophe Hunt}
  \preauthor{\centering\large\emph}
  \postauthor{\par}
  \predate{\centering\large\emph}
  \postdate{\par}
  \date{May 6, 2017}

\usepackage{relsize}
\usepackage{setspace}
\usepackage{amsmath,amsfonts,amsthm}
\usepackage[sfdefault]{roboto}
\usepackage[T1]{fontenc}
\usepackage{float}
\usepackage{multirow}
\usepackage{mathtools}
\usepackage{tikz}

\begin{document}
\maketitle

{
\setcounter{tocdepth}{2}
\tableofcontents
}
\newpage

\section{\texorpdfstring{Write a program to compute the derivative of
\(f(x) = x^3 + 2x^2\) at any value of
\(x\).}{Write a program to compute the derivative of f(x) = x\^{}3 + 2x\^{}2 at any value of x.}}\label{write-a-program-to-compute-the-derivative-of-fx-x3-2x2-at-any-value-of-x.}

\begin{Shaded}
\begin{Highlighting}[]
\NormalTok{deriv_limit <-}\StringTok{ }\NormalTok{function(func, x, h)\{}
               \NormalTok{f <-}\StringTok{ }\NormalTok{function(x) \{}\KeywordTok{eval}\NormalTok{(}\KeywordTok{parse}\NormalTok{(}\DataTypeTok{text =} \NormalTok{func))\}}
               \KeywordTok{return}\NormalTok{((}\KeywordTok{f}\NormalTok{(x +}\StringTok{ }\NormalTok{h) -}\StringTok{ }\KeywordTok{f}\NormalTok{(x)) /}\StringTok{ }\NormalTok{h)}
               \NormalTok{\}}

\KeywordTok{deriv_limit}\NormalTok{(}\DataTypeTok{func =} \NormalTok{(}\StringTok{'x^3 + 2*x^2'}\NormalTok{), }\DataTypeTok{x =} \DecValTok{2}\NormalTok{, }\DataTypeTok{h =} \FloatTok{0.000001}\NormalTok{)}
\end{Highlighting}
\end{Shaded}

\begin{verbatim}
## [1] 20.00001
\end{verbatim}

\begin{Shaded}
\begin{Highlighting}[]
\KeywordTok{deriv_limit}\NormalTok{(}\DataTypeTok{func =} \NormalTok{(}\StringTok{'x^3 + 2*x^2'}\NormalTok{), }\DataTypeTok{x =} \DecValTok{20}\NormalTok{, }\DataTypeTok{h =} \FloatTok{0.000001}\NormalTok{)}
\end{Highlighting}
\end{Shaded}

\begin{verbatim}
## [1] 1280
\end{verbatim}

Test using the analytic form

\begin{Shaded}
\begin{Highlighting}[]
\NormalTok{deriv_analytic <-}\StringTok{ }\NormalTok{function(func, val, var)\{}
                      \NormalTok{f_x <-}\StringTok{ }\KeywordTok{D}\NormalTok{(}\KeywordTok{parse}\NormalTok{(}\DataTypeTok{text =} \NormalTok{func), var) }
                      \KeywordTok{assign}\NormalTok{(var, val)}
                      \KeywordTok{return}\NormalTok{(}\KeywordTok{eval}\NormalTok{(f_x))}
                    \NormalTok{\}}
\KeywordTok{deriv_analytic}\NormalTok{(}\DataTypeTok{func =} \StringTok{'x^3 + 2*x^2'}\NormalTok{, }\DataTypeTok{val =} \DecValTok{2}\NormalTok{, }\DataTypeTok{var =} \StringTok{'x'}\NormalTok{)}
\end{Highlighting}
\end{Shaded}

\begin{verbatim}
## [1] 20
\end{verbatim}

\begin{Shaded}
\begin{Highlighting}[]
\KeywordTok{deriv_analytic}\NormalTok{(}\DataTypeTok{func =} \StringTok{'x^3 + 2*x^2'}\NormalTok{, }\DataTypeTok{val =} \DecValTok{20}\NormalTok{, }\DataTypeTok{var =} \StringTok{'x'}\NormalTok{)}
\end{Highlighting}
\end{Shaded}

\begin{verbatim}
## [1] 1280
\end{verbatim}

Your function should take in a value of x and return back an
approximation to the derivative of \(f(x)\) evaluated at that value. You
should not use the analytical form of the derivative to compute it.
Instead, you should compute this approximation using limits.

\section{\texorpdfstring{Now, write a program to compute the area under
the curve for the function \(3x^2+4x\) in the range x =
\([1, 3]\).}{Now, write a program to compute the area under the curve for the function 3x\^{}2+4x in the range x = {[}1, 3{]}.}}\label{now-write-a-program-to-compute-the-area-under-the-curve-for-the-function-3x24x-in-the-range-x-1-3.}

\begin{Shaded}
\begin{Highlighting}[]
\NormalTok{auc <-}\StringTok{ }\NormalTok{function(func, }\DataTypeTok{range =} \KeywordTok{seq}\NormalTok{(}\DataTypeTok{from =} \DecValTok{1}\NormalTok{, }\DataTypeTok{to =} \DecValTok{3}\NormalTok{, }\DataTypeTok{by =} \FloatTok{0.000001}\NormalTok{))\{}
        \KeywordTok{return}\NormalTok{(}\KeywordTok{sum}\NormalTok{((function(x) \{}\KeywordTok{eval}\NormalTok{(}\KeywordTok{parse}\NormalTok{(}\DataTypeTok{text =} \NormalTok{func))\})(range) *}\StringTok{ }\FloatTok{0.000001}\NormalTok{))}
       \NormalTok{\}}
\KeywordTok{auc}\NormalTok{(}\DataTypeTok{func =} \StringTok{'3*x^2+4*x'}\NormalTok{)}
\end{Highlighting}
\end{Shaded}

\begin{verbatim}
## [1] 42.00002
\end{verbatim}

You should first split the range into many small intervals using some
really small \(\delta x\) value (say 1e-6) and then compute the
approximation to the area under the curve.

\section{Please solve these problems analytically (i.e.~by working out
the math) and submit your
answers.}\label{please-solve-these-problems-analytically-i.e.by-working-out-the-math-and-submit-your-answers.}

\subsection{\texorpdfstring{Use integration by parts to solve for
\(\int sin(x)cos(x)dx\)}{Use integration by parts to solve for \textbackslash{}int sin(x)cos(x)dx}}\label{use-integration-by-parts-to-solve-for-int-sinxcosxdx}

Substitute u = cos(X) and du = -sin(x)dx: \(-\int udu\)\\
u = \(\frac{u^2}{2}\)\\
Therefore: = \(-\frac{u^2}{2} + C\)\\
Substitute back u = cos(x) : = \(-\frac{1}{2}cos^2(x) + C\)

\subsection{\texorpdfstring{Use integration by parts to solve for
\(\int x^2e^xdx\)}{Use integration by parts to solve for \textbackslash{}int x\^{}2e\^{}xdx}}\label{use-integration-by-parts-to-solve-for-int-x2exdx}

For \(e^xx^2\):\\
\(\int fdg = fg - \int gdf\)\\
f = \(x^2\), dg = \(e^xdx\), df = \(2xdx\), g = \(e^x\);
=\(e^x x^2 - 2 \int e^x xdx\)\\
For \(e^xx\): \(\int fdg = fg - \int gdf\) f = \(x\), dg = \(e^xdx\), df
= \(dx\), g = \(e^x\): = \(-2e^xx+e^xx^2+2 \int e^x dx\) The integral of
\(e^x\) is \(e^x\): = \(e^xx^2-2e^xx+2e^x + C\)

\subsection{\texorpdfstring{What is
\(\frac{d}{dx}(x cos(x))\)?}{What is \textbackslash{}frac\{d\}\{dx\}(x cos(x))?}}\label{what-is-fracddxx-cosx}

Use the product rule,
\(\frac{d}{dx}(uv) = v\frac{du}{dx} + u\frac{du}{dv}\) where u =x and v
= \(cos(x)\): = \(cos(x)(\frac{d}{dx}(x))+x(\frac{d}{dx}(cos(x)))\)\\
The derivative of x is 1: = \(cos(x)+x(\frac{d}{dx}(cos(x)))\) The
derivative of cos(x) is -sin(x): = \(cos(x) - sin(x)x\)

\subsection{\texorpdfstring{What is
\(\frac{d}{dx}(e^{x^4})\)?}{What is \textbackslash{}frac\{d\}\{dx\}(e\^{}\{x\^{}4\})?}}\label{what-is-fracddxex4}

Use the chain rule,
\(\frac{d}{dx}(e^{x^4})=\frac{de^u}{du} \frac{du}{dx}\), where \(u=x^4\)
and \(\frac{d}{du}(e^u)=e^u\) =\(e^{x^4}(\frac{d}{du}(x^4))\) Use the
power rule, \(\frac{d}{dx}(x^n)=nx^{n-1}\), where n = 4:
\(\frac{d}{dx}(x^4)=4x^3\): \(= 4x^3e^{x^4}\)


\end{document}
