\documentclass[]{article}
\usepackage{lmodern}
\usepackage{amssymb,amsmath}
\usepackage{ifxetex,ifluatex}
\usepackage{fixltx2e} % provides \textsubscript
\ifnum 0\ifxetex 1\fi\ifluatex 1\fi=0 % if pdftex
  \usepackage[T1]{fontenc}
  \usepackage[utf8]{inputenc}
\else % if luatex or xelatex
  \ifxetex
    \usepackage{mathspec}
  \else
    \usepackage{fontspec}
  \fi
  \defaultfontfeatures{Ligatures=TeX,Scale=MatchLowercase}
\fi
% use upquote if available, for straight quotes in verbatim environments
\IfFileExists{upquote.sty}{\usepackage{upquote}}{}
% use microtype if available
\IfFileExists{microtype.sty}{%
\usepackage{microtype}
\UseMicrotypeSet[protrusion]{basicmath} % disable protrusion for tt fonts
}{}
\usepackage[margin=1in]{geometry}
\usepackage{hyperref}
\hypersetup{unicode=true,
            pdftitle={Homework 9},
            pdfauthor={Christophe Hunt},
            pdfborder={0 0 0},
            breaklinks=true}
\urlstyle{same}  % don't use monospace font for urls
\usepackage{graphicx,grffile}
\makeatletter
\def\maxwidth{\ifdim\Gin@nat@width>\linewidth\linewidth\else\Gin@nat@width\fi}
\def\maxheight{\ifdim\Gin@nat@height>\textheight\textheight\else\Gin@nat@height\fi}
\makeatother
% Scale images if necessary, so that they will not overflow the page
% margins by default, and it is still possible to overwrite the defaults
% using explicit options in \includegraphics[width, height, ...]{}
\setkeys{Gin}{width=\maxwidth,height=\maxheight,keepaspectratio}
\IfFileExists{parskip.sty}{%
\usepackage{parskip}
}{% else
\setlength{\parindent}{0pt}
\setlength{\parskip}{6pt plus 2pt minus 1pt}
}
\setlength{\emergencystretch}{3em}  % prevent overfull lines
\providecommand{\tightlist}{%
  \setlength{\itemsep}{0pt}\setlength{\parskip}{0pt}}
\setcounter{secnumdepth}{5}
% Redefines (sub)paragraphs to behave more like sections
\ifx\paragraph\undefined\else
\let\oldparagraph\paragraph
\renewcommand{\paragraph}[1]{\oldparagraph{#1}\mbox{}}
\fi
\ifx\subparagraph\undefined\else
\let\oldsubparagraph\subparagraph
\renewcommand{\subparagraph}[1]{\oldsubparagraph{#1}\mbox{}}
\fi

%%% Use protect on footnotes to avoid problems with footnotes in titles
\let\rmarkdownfootnote\footnote%
\def\footnote{\protect\rmarkdownfootnote}

%%% Change title format to be more compact
\usepackage{titling}

% Create subtitle command for use in maketitle
\newcommand{\subtitle}[1]{
  \posttitle{
    \begin{center}\large#1\end{center}
    }
}

\setlength{\droptitle}{-2em}
  \title{Homework 9}
  \pretitle{\vspace{\droptitle}\centering\huge}
  \posttitle{\par}
  \author{Christophe Hunt}
  \preauthor{\centering\large\emph}
  \postauthor{\par}
  \predate{\centering\large\emph}
  \postdate{\par}
  \date{April 1, 2017}

\usepackage{relsize}
\usepackage{setspace}
\usepackage{amsmath,amsfonts,amsthm}
\usepackage[sfdefault]{roboto}
\usepackage[T1]{fontenc}
\usepackage{float}
\usepackage{multirow}
\usepackage{mathtools}
\usepackage{tikz}

\begin{document}
\maketitle

{
\setcounter{tocdepth}{2}
\tableofcontents
}
This week, we'll empirically verify Central Limit Theorem. We'll write
code to run a small simulation on some distributions and verify that the
results match what we expect from Central Limit Theorem. Please use R
markdown to capture all your experiments and code. Please submit your
Rmd file with your name as the filename.

\section{(1) First write a function that will produce a sample of random
variable that is distributed as
follows:}\label{first-write-a-function-that-will-produce-a-sample-of-random-variable-that-is-distributed-as-follows}

\subsection{TODO - check these formulas might have not copied
correctly}\label{todo---check-these-formulas-might-have-not-copied-correctly}

\(f(x) = x, 0 \leq x \leq 1\) \(f(x) = 2 > x, 1 < x \leq 2\)

That is, when your function is called, it will return a random variable
between 0 and 2 that is distributed according to the above PDF. Please
note that this is not the same as writing a function and sampling
uniformly from it. In the online session this week, I'll cover Sampling
techniques. You will find it useful when you do the assignment for this
week. In addition, as usual, there are one-liners in R that will give
you samples from a function. We'll cover both of these approaches in the
online session.

\section{(2) Now, write a function that will produce a sample of random
variable that is distributed as
follows:}\label{now-write-a-function-that-will-produce-a-sample-of-random-variable-that-is-distributed-as-follows}

\subsection{TODO - check these formulas might have not copied
correctly}\label{todo---check-these-formulas-might-have-not-copied-correctly-1}

\(f(x) = 1 > x, 0 \leq x \leq 1\) \(f(x) = x > 1, 1 < x \leq 2 (2)\)

\section{(3) Draw 1000 samples (call your function 1000 times each) from
each of the above two distributions and plot the resulting histograms.
You should have one histogram for each PDF. See that it matches your
understanding of these
PDFs.}\label{draw-1000-samples-call-your-function-1000-times-each-from-each-of-the-above-two-distributions-and-plot-the-resulting-histograms.-you-should-have-one-histogram-for-each-pdf.-see-that-it-matches-your-understanding-of-these-pdfs.}

\section{(4) Now, write a program that will take a sample set size n as
a parameter and the PDF as the second parameter, and perform 1000
iterations where it samples from the PDF, each time taking n samples and
computes the mean of these n samples. It then plots a histogram of these
1000 means that it
computes.}\label{now-write-a-program-that-will-take-a-sample-set-size-n-as-a-parameter-and-the-pdf-as-the-second-parameter-and-perform-1000-iterations-where-it-samples-from-the-pdf-each-time-taking-n-samples-and-computes-the-mean-of-these-n-samples.-it-then-plots-a-histogram-of-these-1000-means-that-it-computes.}

\section{(5) Verify that as you set n to something like 10 or 20, each
of the two PDFs produce normally distributed mean of samples,
empirically verifying the Central Limit Theorem. Please play around with
various values of n and you'll see that even for reasonably small sample
sizes such as 10, Central Limit Theorem
holds.}\label{verify-that-as-you-set-n-to-something-like-10-or-20-each-of-the-two-pdfs-produce-normally-distributed-mean-of-samples-empirically-verifying-the-central-limit-theorem.-please-play-around-with-various-values-of-n-and-youll-see-that-even-for-reasonably-small-sample-sizes-such-as-10-central-limit-theorem-holds.}


\end{document}
