\documentclass[]{article}
\usepackage{lmodern}
\usepackage{amssymb,amsmath}
\usepackage{ifxetex,ifluatex}
\usepackage{fixltx2e} % provides \textsubscript
\ifnum 0\ifxetex 1\fi\ifluatex 1\fi=0 % if pdftex
  \usepackage[T1]{fontenc}
  \usepackage[utf8]{inputenc}
\else % if luatex or xelatex
  \ifxetex
    \usepackage{mathspec}
  \else
    \usepackage{fontspec}
  \fi
  \defaultfontfeatures{Ligatures=TeX,Scale=MatchLowercase}
\fi
% use upquote if available, for straight quotes in verbatim environments
\IfFileExists{upquote.sty}{\usepackage{upquote}}{}
% use microtype if available
\IfFileExists{microtype.sty}{%
\usepackage{microtype}
\UseMicrotypeSet[protrusion]{basicmath} % disable protrusion for tt fonts
}{}
\usepackage[margin=1in]{geometry}
\usepackage{hyperref}
\hypersetup{unicode=true,
            pdftitle={Homework 3},
            pdfauthor={Christophe Hunt},
            pdfborder={0 0 0},
            breaklinks=true}
\urlstyle{same}  % don't use monospace font for urls
\usepackage{color}
\usepackage{fancyvrb}
\newcommand{\VerbBar}{|}
\newcommand{\VERB}{\Verb[commandchars=\\\{\}]}
\DefineVerbatimEnvironment{Highlighting}{Verbatim}{commandchars=\\\{\}}
% Add ',fontsize=\small' for more characters per line
\usepackage{framed}
\definecolor{shadecolor}{RGB}{248,248,248}
\newenvironment{Shaded}{\begin{snugshade}}{\end{snugshade}}
\newcommand{\KeywordTok}[1]{\textcolor[rgb]{0.13,0.29,0.53}{\textbf{{#1}}}}
\newcommand{\DataTypeTok}[1]{\textcolor[rgb]{0.13,0.29,0.53}{{#1}}}
\newcommand{\DecValTok}[1]{\textcolor[rgb]{0.00,0.00,0.81}{{#1}}}
\newcommand{\BaseNTok}[1]{\textcolor[rgb]{0.00,0.00,0.81}{{#1}}}
\newcommand{\FloatTok}[1]{\textcolor[rgb]{0.00,0.00,0.81}{{#1}}}
\newcommand{\ConstantTok}[1]{\textcolor[rgb]{0.00,0.00,0.00}{{#1}}}
\newcommand{\CharTok}[1]{\textcolor[rgb]{0.31,0.60,0.02}{{#1}}}
\newcommand{\SpecialCharTok}[1]{\textcolor[rgb]{0.00,0.00,0.00}{{#1}}}
\newcommand{\StringTok}[1]{\textcolor[rgb]{0.31,0.60,0.02}{{#1}}}
\newcommand{\VerbatimStringTok}[1]{\textcolor[rgb]{0.31,0.60,0.02}{{#1}}}
\newcommand{\SpecialStringTok}[1]{\textcolor[rgb]{0.31,0.60,0.02}{{#1}}}
\newcommand{\ImportTok}[1]{{#1}}
\newcommand{\CommentTok}[1]{\textcolor[rgb]{0.56,0.35,0.01}{\textit{{#1}}}}
\newcommand{\DocumentationTok}[1]{\textcolor[rgb]{0.56,0.35,0.01}{\textbf{\textit{{#1}}}}}
\newcommand{\AnnotationTok}[1]{\textcolor[rgb]{0.56,0.35,0.01}{\textbf{\textit{{#1}}}}}
\newcommand{\CommentVarTok}[1]{\textcolor[rgb]{0.56,0.35,0.01}{\textbf{\textit{{#1}}}}}
\newcommand{\OtherTok}[1]{\textcolor[rgb]{0.56,0.35,0.01}{{#1}}}
\newcommand{\FunctionTok}[1]{\textcolor[rgb]{0.00,0.00,0.00}{{#1}}}
\newcommand{\VariableTok}[1]{\textcolor[rgb]{0.00,0.00,0.00}{{#1}}}
\newcommand{\ControlFlowTok}[1]{\textcolor[rgb]{0.13,0.29,0.53}{\textbf{{#1}}}}
\newcommand{\OperatorTok}[1]{\textcolor[rgb]{0.81,0.36,0.00}{\textbf{{#1}}}}
\newcommand{\BuiltInTok}[1]{{#1}}
\newcommand{\ExtensionTok}[1]{{#1}}
\newcommand{\PreprocessorTok}[1]{\textcolor[rgb]{0.56,0.35,0.01}{\textit{{#1}}}}
\newcommand{\AttributeTok}[1]{\textcolor[rgb]{0.77,0.63,0.00}{{#1}}}
\newcommand{\RegionMarkerTok}[1]{{#1}}
\newcommand{\InformationTok}[1]{\textcolor[rgb]{0.56,0.35,0.01}{\textbf{\textit{{#1}}}}}
\newcommand{\WarningTok}[1]{\textcolor[rgb]{0.56,0.35,0.01}{\textbf{\textit{{#1}}}}}
\newcommand{\AlertTok}[1]{\textcolor[rgb]{0.94,0.16,0.16}{{#1}}}
\newcommand{\ErrorTok}[1]{\textcolor[rgb]{0.64,0.00,0.00}{\textbf{{#1}}}}
\newcommand{\NormalTok}[1]{{#1}}
\usepackage{graphicx,grffile}
\makeatletter
\def\maxwidth{\ifdim\Gin@nat@width>\linewidth\linewidth\else\Gin@nat@width\fi}
\def\maxheight{\ifdim\Gin@nat@height>\textheight\textheight\else\Gin@nat@height\fi}
\makeatother
% Scale images if necessary, so that they will not overflow the page
% margins by default, and it is still possible to overwrite the defaults
% using explicit options in \includegraphics[width, height, ...]{}
\setkeys{Gin}{width=\maxwidth,height=\maxheight,keepaspectratio}
\IfFileExists{parskip.sty}{%
\usepackage{parskip}
}{% else
\setlength{\parindent}{0pt}
\setlength{\parskip}{6pt plus 2pt minus 1pt}
}
\setlength{\emergencystretch}{3em}  % prevent overfull lines
\providecommand{\tightlist}{%
  \setlength{\itemsep}{0pt}\setlength{\parskip}{0pt}}
\setcounter{secnumdepth}{5}
% Redefines (sub)paragraphs to behave more like sections
\ifx\paragraph\undefined\else
\let\oldparagraph\paragraph
\renewcommand{\paragraph}[1]{\oldparagraph{#1}\mbox{}}
\fi
\ifx\subparagraph\undefined\else
\let\oldsubparagraph\subparagraph
\renewcommand{\subparagraph}[1]{\oldsubparagraph{#1}\mbox{}}
\fi

%%% Use protect on footnotes to avoid problems with footnotes in titles
\let\rmarkdownfootnote\footnote%
\def\footnote{\protect\rmarkdownfootnote}

%%% Change title format to be more compact
\usepackage{titling}

% Create subtitle command for use in maketitle
\newcommand{\subtitle}[1]{
  \posttitle{
    \begin{center}\large#1\end{center}
    }
}

\setlength{\droptitle}{-2em}
  \title{Homework 3}
  \pretitle{\vspace{\droptitle}\centering\huge}
  \posttitle{\par}
  \author{Christophe Hunt}
  \preauthor{\centering\large\emph}
  \postauthor{\par}
  \predate{\centering\large\emph}
  \postdate{\par}
  \date{February 15, 2017}

\usepackage{relsize}
\usepackage{setspace}
\usepackage{amsmath,amsfonts,amsthm}
\usepackage[sfdefault]{roboto}
\usepackage[T1]{fontenc}
\usepackage{float}

\begin{document}
\maketitle

{
\setcounter{tocdepth}{2}
\tableofcontents
}
\section{Problem Set 1}\label{problem-set-1}

\subsection{Problem 1}\label{problem-1}

What is the rank of the matrix A?

\[
A = \begin{bmatrix}
    1 & 2 & 3 &4  \\
   -1 & 0 & 1 & 3  \\
   0 & 1 & -2 & 1   \\
   5 & 4 & -2 & -3   \\
\end{bmatrix}
\]

\begin{Shaded}
\begin{Highlighting}[]
\NormalTok{A <-}\StringTok{ }\KeywordTok{t}\NormalTok{(}\KeywordTok{matrix}\NormalTok{(}\KeywordTok{c}\NormalTok{(}\DecValTok{1} \NormalTok{, }\DecValTok{2} \NormalTok{,}\DecValTok{3} \NormalTok{, }\DecValTok{4}\NormalTok{, }
              \NormalTok{-}\DecValTok{1} \NormalTok{, }\DecValTok{0} \NormalTok{, }\DecValTok{1} \NormalTok{, }\DecValTok{3}\NormalTok{, }
              \DecValTok{0} \NormalTok{, }\DecValTok{1} \NormalTok{, -}\DecValTok{2} \NormalTok{, }\DecValTok{1}\NormalTok{, }
              \DecValTok{5} \NormalTok{, }\DecValTok{4} \NormalTok{, -}\DecValTok{2} \NormalTok{, -}\DecValTok{3}\NormalTok{), }\DataTypeTok{nrow =} \DecValTok{4}\NormalTok{, }\DataTypeTok{ncol =} \DecValTok{4}\NormalTok{))}
\end{Highlighting}
\end{Shaded}

Since A is a square matrix of (4 x 4) and the determinate is -9 which is
\(\neq\) 0, the rank is simply 4

Below is my attempt to create a function to rank matrix A and matrix B
in problem 3 programmatically. I think that the part of the function
that calculates the determinates of the submatrices could be improved.

\begin{Shaded}
\begin{Highlighting}[]
\NormalTok{A <-}\StringTok{ }\KeywordTok{t}\NormalTok{(}\KeywordTok{matrix}\NormalTok{(}\KeywordTok{c}\NormalTok{(}\DecValTok{1} \NormalTok{, }\DecValTok{2} \NormalTok{,}\DecValTok{3} \NormalTok{, }\DecValTok{4}\NormalTok{, }
              \NormalTok{-}\DecValTok{1} \NormalTok{, }\DecValTok{0} \NormalTok{, }\DecValTok{1} \NormalTok{, }\DecValTok{3}\NormalTok{, }
              \DecValTok{0} \NormalTok{, }\DecValTok{1} \NormalTok{, -}\DecValTok{2} \NormalTok{, }\DecValTok{1}\NormalTok{, }
              \DecValTok{5} \NormalTok{, }\DecValTok{4} \NormalTok{, -}\DecValTok{2} \NormalTok{, -}\DecValTok{3}\NormalTok{), }\DataTypeTok{nrow =} \DecValTok{4}\NormalTok{, }\DataTypeTok{ncol =} \DecValTok{4}\NormalTok{))}

\NormalTok{subdet <-}\StringTok{ }\NormalTok{function(A, t)\{}
            \NormalTok{subdet <-}\StringTok{ }\DecValTok{0}
           \NormalTok{for (i in }\DecValTok{1}\NormalTok{:t)\{}
             \NormalTok{subdet <-}\StringTok{ }\KeywordTok{rbind}\NormalTok{(subdet,}\KeywordTok{det}\NormalTok{(A[-i,-i]))}
           \NormalTok{\}}
           \NormalTok{if (}\KeywordTok{sum}\NormalTok{(subdet) !=}\DecValTok{0}\NormalTok{)\{}
             \KeywordTok{return}\NormalTok{(}\KeywordTok{as.integer}\NormalTok{(}\KeywordTok{ncol}\NormalTok{(A))-}\DecValTok{1}\NormalTok{)}
           \NormalTok{\} else \{}
             \DecValTok{0}
           \NormalTok{\}}
\NormalTok{\}}

\NormalTok{Matrix.Rank <-}\StringTok{ }\NormalTok{function(A)\{}
               \NormalTok{sq.matrix <-}\StringTok{ }\NormalTok{(}\KeywordTok{as.integer}\NormalTok{(}\KeywordTok{ncol}\NormalTok{(A)) ==}\StringTok{ }\KeywordTok{as.integer}\NormalTok{(}\KeywordTok{nrow}\NormalTok{(A)))}
               \NormalTok{if (}\KeywordTok{all}\NormalTok{(A ==}\StringTok{ }\DecValTok{0}\NormalTok{))\{}
                 \KeywordTok{return}\NormalTok{(}\DecValTok{0}\NormalTok{)}
               \NormalTok{\} else if (sq.matrix ==}\StringTok{ }\OtherTok{TRUE}\NormalTok{)\{}
                 \NormalTok{det}\FloatTok{.0} \NormalTok{<-}\StringTok{ }\KeywordTok{det}\NormalTok{(A) !=}\StringTok{ }\DecValTok{0}
                 \NormalTok{if (det}\FloatTok{.0} \NormalTok{==}\StringTok{ }\OtherTok{TRUE}\NormalTok{) \{}
                 \KeywordTok{return}\NormalTok{(}\KeywordTok{as.integer}\NormalTok{(}\KeywordTok{ncol}\NormalTok{(A)))}
                 \NormalTok{\} else if (}\KeywordTok{subdet}\NormalTok{(A,}\KeywordTok{ncol}\NormalTok{(A)) !=}\StringTok{ }\DecValTok{0}\NormalTok{) \{}
                  \KeywordTok{return}\NormalTok{(}\KeywordTok{as.integer}\NormalTok{(}\KeywordTok{ncol}\NormalTok{(A))-}\DecValTok{1}\NormalTok{)}
                 \NormalTok{\} else \{}
                     \KeywordTok{return}\NormalTok{(}\DecValTok{1}\NormalTok{)}
                   \NormalTok{\}}
                 \NormalTok{\}}
               \NormalTok{\}}

\KeywordTok{Matrix.Rank}\NormalTok{(A)}
\end{Highlighting}
\end{Shaded}

\begin{verbatim}
## [1] 4
\end{verbatim}

\subsection{Problem 2}\label{problem-2}

Given an mxn matrix where \(m > n\), what can be the maximum rank? The
minimum rank, assuming that the matrix is non-zero?

\begin{quote}
The maximum rank of a rectangular matrix is the maximum columns or rows
for the lesser value. Therefore, given an mxn matrix where \(m > n\),
the maximum rank is \(n\).
\end{quote}

\begin{quote}
Assuming the rectangular matrix has at least one non-zero element, it's
minimum rank must be greater than zero, therefore the minimum rank would
be 1.
\end{quote}

\subsection{Problem 3}\label{problem-3}

What is the rank of matrix B?

\[
B = 
\begin{bmatrix}
   1 & 2 & 1 \\
   3 & 6 & 3 \\
   2 & 4 & 2 \\
\end{bmatrix}
\]

\begin{Shaded}
\begin{Highlighting}[]
\NormalTok{B <-}\StringTok{ }\KeywordTok{t}\NormalTok{(}\KeywordTok{matrix}\NormalTok{(}\KeywordTok{c}\NormalTok{( }\DecValTok{1} \NormalTok{, }\DecValTok{2} \NormalTok{, }\DecValTok{1} \NormalTok{,  }\DecValTok{3} \NormalTok{,  }\DecValTok{6} \NormalTok{, }\DecValTok{3} \NormalTok{,  }\DecValTok{2} \NormalTok{,  }\DecValTok{4} \NormalTok{, }\DecValTok{2} \NormalTok{), }\DataTypeTok{ncol =} \DecValTok{3}\NormalTok{, }\DataTypeTok{nrow =} \DecValTok{3}\NormalTok{))}
\end{Highlighting}
\end{Shaded}

The matrix rows are linearly dependent, as R2 = 3, 6, 3 and R3 = 2, 4, 2
are mulitples of R1 = 1, 2, 1. This made my function more challenging
because the determinates of the submatrices of B are also \(= 0\).
However, as long as there is at least one non-zero element in the matrix
the minimum rank will be \(= 1\).

\begin{Shaded}
\begin{Highlighting}[]
\KeywordTok{Matrix.Rank}\NormalTok{(B)}
\end{Highlighting}
\end{Shaded}

\begin{verbatim}
## [1] 1
\end{verbatim}

\section{Problem Set 2}\label{problem-set-2}

Compute the eigenvalues and eigenvectors of the matrix A. You'll need to
show your work. You'll need to write out the characteristic polynomial
and show your solution.

\[
A = 
\begin{bmatrix}
   1 & 2 & 3 \\
   0 & 4 & 5 \\
   0 & 0 & 6 \\
\end{bmatrix}
\]

\[
det( 
\begin{bmatrix}
   1 & 2 & 3 \\
   0 & 4 & 5 \\
   0 & 0 & 6 \\
\end{bmatrix}
-
\begin{bmatrix}
   \lambda & 0 & 0 \\
   0 & \lambda  & 0 \\
   0 & 0 & \lambda  \\
\end{bmatrix}
) = 0
\]

\[
det( 
\begin{bmatrix}
   1 -\lambda  & 2 & 3 \\
   0 &  4 -\lambda& 5 \\
   0 & 0 & 6 -\lambda \\
\end{bmatrix}
)=0
\]

which reduces to:

\((1-\lambda)(4-\lambda)(6-\lambda) =0\)

Therefore, our Eigen Values are:

\(\lambda_1 = 1 ; \lambda_2 = 4; \lambda_3 = 6\)

Our Eigen Vectors are as follows:

\(\lambda_1 = 1\)

\[
det( 
\begin{bmatrix}
   1 & 2 & 3 \\
   0 & 4 & 5 \\
   0 & 0 & 6 \\
\end{bmatrix}
-
\begin{bmatrix}
   1 & 0 & 0 \\
   0 & 1  & 0 \\
   0 & 0 & 1 \\
\end{bmatrix}
) = 0
\]

\[
\begin{bmatrix}
   0 & 2 & 3 \\
   0 & 3 & 5 \\
   0 & 0 & 5 \\
\end{bmatrix}
\begin{bmatrix}
v_1 \\
v_2 \\
v_3
\end{bmatrix}
=0
\]

\[v_1 = 1;~v_2 = 0;~v_3 = 0\]

Therefore:

\[
E\lambda_{=1} = \
\begin{bmatrix}
1 \\
0 \\
0
\end{bmatrix}
\]

\begin{center}\rule{0.5\linewidth}{\linethickness}\end{center}

\newpage

\[\lambda_2 = 4\]

\[
det( 
\begin{bmatrix}
   1 & 2 & 3 \\
   0 & 4 & 5 \\
   0 & 0 & 6 \\
\end{bmatrix}
-
\begin{bmatrix}
   4 & 0 & 0 \\
   0 & 4  & 0 \\
   0 & 0 & 4 \\
\end{bmatrix}
) = 0
\]

\[
\begin{bmatrix}
   -3 & 2 & 3 \\
   0 & 0 & 5 \\
   0 & 0 & 2 \\
\end{bmatrix}
\begin{bmatrix}
v_1 \\
v_2 \\
v_3
\end{bmatrix}
=0
\]

Which we can reduce to \(-3v_1+2v_2+3v_3 = 0~and~v_3 =0\); substituting
back we have \(-3v_1+2v_2=0\) or \(3v_1=2v_2\) then
\(v_1=\frac{3}{2}v_2\) where \(v_1 =1\).

Therefore:

\[
E\lambda_{=4} = \
\begin{bmatrix}
1 \\
\frac{3}{2} \\
0
\end{bmatrix}
\]

\begin{center}\rule{0.5\linewidth}{\linethickness}\end{center}

\[\lambda_3 = 6\]

\[
det( 
\begin{bmatrix}
   1 & 2 & 3 \\
   0 & 4 & 5 \\
   0 & 0 & 6 \\
\end{bmatrix}
-
\begin{bmatrix}
   6 & 0 & 0 \\
   0 & 6 & 0 \\
   0 & 0 & 6 \\
\end{bmatrix}
) = 0
\]

\[
\begin{bmatrix}
   -5 & 2 & 3 \\
   0 & -2 & 5 \\
   0 & 0 & 0 \\
\end{bmatrix}
\begin{bmatrix}
v_1 \\
v_2 \\
v_3
\end{bmatrix}
=0
\]

Which we can reduce to \(-5v_1+2v_2+3v_3 = 0~and~-2v_1+5v_2=0\); solving
for \(-2v_2+5v_3=0\), we reduce to \(5v_3=2v_2\) or
\(v_3=\frac{5}{2}v_2\). Substituting back \(v_3=1\) we get
\(1=\frac{5}{2}v_2\) or \(\frac{2}{5}=v_2\). Now, subsituting back to
\(-5v_1+2v_2+3v_3 = 0\) or \(-5v_1+2(\frac{2}{5})+3(1) = 0\) then
\(-5v_1+3\frac{4}{5}= 0\) then \(5v_1= 3\frac{4}{5}\) then
\(5v_1= 3\frac{4}{5}\) then \(v_1= \frac{19}{25}\)

Therefore:

\[
E\lambda_{=6} = \
\begin{bmatrix}
\frac{19}{25} \\
\frac{2}{5} \\
1
\end{bmatrix}
\]


\end{document}
