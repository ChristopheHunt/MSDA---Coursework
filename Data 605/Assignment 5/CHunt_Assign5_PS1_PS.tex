\documentclass[]{article}
\usepackage{lmodern}
\usepackage{amssymb,amsmath}
\usepackage{ifxetex,ifluatex}
\usepackage{fixltx2e} % provides \textsubscript
\ifnum 0\ifxetex 1\fi\ifluatex 1\fi=0 % if pdftex
  \usepackage[T1]{fontenc}
  \usepackage[utf8]{inputenc}
\else % if luatex or xelatex
  \ifxetex
    \usepackage{mathspec}
  \else
    \usepackage{fontspec}
  \fi
  \defaultfontfeatures{Ligatures=TeX,Scale=MatchLowercase}
\fi
% use upquote if available, for straight quotes in verbatim environments
\IfFileExists{upquote.sty}{\usepackage{upquote}}{}
% use microtype if available
\IfFileExists{microtype.sty}{%
\usepackage{microtype}
\UseMicrotypeSet[protrusion]{basicmath} % disable protrusion for tt fonts
}{}
\usepackage[margin=1in]{geometry}
\usepackage{hyperref}
\hypersetup{unicode=true,
            pdftitle={Homework 5},
            pdfauthor={Christophe Hunt},
            pdfborder={0 0 0},
            breaklinks=true}
\urlstyle{same}  % don't use monospace font for urls
\usepackage{color}
\usepackage{fancyvrb}
\newcommand{\VerbBar}{|}
\newcommand{\VERB}{\Verb[commandchars=\\\{\}]}
\DefineVerbatimEnvironment{Highlighting}{Verbatim}{commandchars=\\\{\}}
% Add ',fontsize=\small' for more characters per line
\usepackage{framed}
\definecolor{shadecolor}{RGB}{248,248,248}
\newenvironment{Shaded}{\begin{snugshade}}{\end{snugshade}}
\newcommand{\KeywordTok}[1]{\textcolor[rgb]{0.13,0.29,0.53}{\textbf{{#1}}}}
\newcommand{\DataTypeTok}[1]{\textcolor[rgb]{0.13,0.29,0.53}{{#1}}}
\newcommand{\DecValTok}[1]{\textcolor[rgb]{0.00,0.00,0.81}{{#1}}}
\newcommand{\BaseNTok}[1]{\textcolor[rgb]{0.00,0.00,0.81}{{#1}}}
\newcommand{\FloatTok}[1]{\textcolor[rgb]{0.00,0.00,0.81}{{#1}}}
\newcommand{\ConstantTok}[1]{\textcolor[rgb]{0.00,0.00,0.00}{{#1}}}
\newcommand{\CharTok}[1]{\textcolor[rgb]{0.31,0.60,0.02}{{#1}}}
\newcommand{\SpecialCharTok}[1]{\textcolor[rgb]{0.00,0.00,0.00}{{#1}}}
\newcommand{\StringTok}[1]{\textcolor[rgb]{0.31,0.60,0.02}{{#1}}}
\newcommand{\VerbatimStringTok}[1]{\textcolor[rgb]{0.31,0.60,0.02}{{#1}}}
\newcommand{\SpecialStringTok}[1]{\textcolor[rgb]{0.31,0.60,0.02}{{#1}}}
\newcommand{\ImportTok}[1]{{#1}}
\newcommand{\CommentTok}[1]{\textcolor[rgb]{0.56,0.35,0.01}{\textit{{#1}}}}
\newcommand{\DocumentationTok}[1]{\textcolor[rgb]{0.56,0.35,0.01}{\textbf{\textit{{#1}}}}}
\newcommand{\AnnotationTok}[1]{\textcolor[rgb]{0.56,0.35,0.01}{\textbf{\textit{{#1}}}}}
\newcommand{\CommentVarTok}[1]{\textcolor[rgb]{0.56,0.35,0.01}{\textbf{\textit{{#1}}}}}
\newcommand{\OtherTok}[1]{\textcolor[rgb]{0.56,0.35,0.01}{{#1}}}
\newcommand{\FunctionTok}[1]{\textcolor[rgb]{0.00,0.00,0.00}{{#1}}}
\newcommand{\VariableTok}[1]{\textcolor[rgb]{0.00,0.00,0.00}{{#1}}}
\newcommand{\ControlFlowTok}[1]{\textcolor[rgb]{0.13,0.29,0.53}{\textbf{{#1}}}}
\newcommand{\OperatorTok}[1]{\textcolor[rgb]{0.81,0.36,0.00}{\textbf{{#1}}}}
\newcommand{\BuiltInTok}[1]{{#1}}
\newcommand{\ExtensionTok}[1]{{#1}}
\newcommand{\PreprocessorTok}[1]{\textcolor[rgb]{0.56,0.35,0.01}{\textit{{#1}}}}
\newcommand{\AttributeTok}[1]{\textcolor[rgb]{0.77,0.63,0.00}{{#1}}}
\newcommand{\RegionMarkerTok}[1]{{#1}}
\newcommand{\InformationTok}[1]{\textcolor[rgb]{0.56,0.35,0.01}{\textbf{\textit{{#1}}}}}
\newcommand{\WarningTok}[1]{\textcolor[rgb]{0.56,0.35,0.01}{\textbf{\textit{{#1}}}}}
\newcommand{\AlertTok}[1]{\textcolor[rgb]{0.94,0.16,0.16}{{#1}}}
\newcommand{\ErrorTok}[1]{\textcolor[rgb]{0.64,0.00,0.00}{\textbf{{#1}}}}
\newcommand{\NormalTok}[1]{{#1}}
\usepackage{graphicx,grffile}
\makeatletter
\def\maxwidth{\ifdim\Gin@nat@width>\linewidth\linewidth\else\Gin@nat@width\fi}
\def\maxheight{\ifdim\Gin@nat@height>\textheight\textheight\else\Gin@nat@height\fi}
\makeatother
% Scale images if necessary, so that they will not overflow the page
% margins by default, and it is still possible to overwrite the defaults
% using explicit options in \includegraphics[width, height, ...]{}
\setkeys{Gin}{width=\maxwidth,height=\maxheight,keepaspectratio}
\IfFileExists{parskip.sty}{%
\usepackage{parskip}
}{% else
\setlength{\parindent}{0pt}
\setlength{\parskip}{6pt plus 2pt minus 1pt}
}
\setlength{\emergencystretch}{3em}  % prevent overfull lines
\providecommand{\tightlist}{%
  \setlength{\itemsep}{0pt}\setlength{\parskip}{0pt}}
\setcounter{secnumdepth}{5}
% Redefines (sub)paragraphs to behave more like sections
\ifx\paragraph\undefined\else
\let\oldparagraph\paragraph
\renewcommand{\paragraph}[1]{\oldparagraph{#1}\mbox{}}
\fi
\ifx\subparagraph\undefined\else
\let\oldsubparagraph\subparagraph
\renewcommand{\subparagraph}[1]{\oldsubparagraph{#1}\mbox{}}
\fi

%%% Use protect on footnotes to avoid problems with footnotes in titles
\let\rmarkdownfootnote\footnote%
\def\footnote{\protect\rmarkdownfootnote}

%%% Change title format to be more compact
\usepackage{titling}

% Create subtitle command for use in maketitle
\newcommand{\subtitle}[1]{
  \posttitle{
    \begin{center}\large#1\end{center}
    }
}

\setlength{\droptitle}{-2em}
  \title{Homework 5}
  \pretitle{\vspace{\droptitle}\centering\huge}
  \posttitle{\par}
  \author{Christophe Hunt}
  \preauthor{\centering\large\emph}
  \postauthor{\par}
  \predate{\centering\large\emph}
  \postdate{\par}
  \date{March 4, 2017}

\usepackage{relsize}
\usepackage{setspace}
\usepackage{amsmath,amsfonts,amsthm}
\usepackage[sfdefault]{roboto}
\usepackage[T1]{fontenc}
\usepackage{float}

\begin{document}
\maketitle

{
\setcounter{tocdepth}{2}
\tableofcontents
}
\section{Problem Set 1}\label{problem-set-1}

Consider the unsolvable system Ax = b as given below: \[
\begin{bmatrix}
    1 & 0  \\
    1 & 1 \\
    1 & 3 \\
    1 & 4 \\
\end{bmatrix}
\begin{bmatrix}
x_1 \\
x_2 \\
\end{bmatrix}
=
\begin{bmatrix}
0 \\
8 \\
8 \\
20  \\
\end{bmatrix}
\]

\subsection{\texorpdfstring{Write R Markdown script to compute \(A^TA\)
and
\(A^Tb\)}{Write R Markdown script to compute A\^{}TA and A\^{}Tb}}\label{write-r-markdown-script-to-compute-ata-and-atb}

\begin{Shaded}
\begin{Highlighting}[]
\NormalTok{A <-}\StringTok{ }\KeywordTok{matrix}\NormalTok{(}\KeywordTok{c}\NormalTok{(}\DecValTok{1}\NormalTok{,}\DecValTok{1}\NormalTok{,}\DecValTok{1}\NormalTok{,}\DecValTok{1}\NormalTok{,}\DecValTok{0}\NormalTok{,}\DecValTok{1}\NormalTok{,}\DecValTok{3}\NormalTok{,}\DecValTok{4}\NormalTok{), }\DataTypeTok{ncol =} \DecValTok{2}\NormalTok{)}
\NormalTok{b <-}\StringTok{ }\KeywordTok{matrix}\NormalTok{(}\KeywordTok{c}\NormalTok{(}\DecValTok{0}\NormalTok{,}\DecValTok{8}\NormalTok{,}\DecValTok{8}\NormalTok{,}\DecValTok{20}\NormalTok{))}

\NormalTok{ATA <-}\StringTok{ }\KeywordTok{t}\NormalTok{(A) %*%}\StringTok{ }\NormalTok{A}
\NormalTok{ATb <-}\StringTok{ }\KeywordTok{t}\NormalTok{(A) %*%}\StringTok{ }\NormalTok{b}

\NormalTok{results <-}\StringTok{ }\KeywordTok{list}\NormalTok{(}\StringTok{"ATA"} \NormalTok{=}\StringTok{ }\NormalTok{ATA, }\StringTok{"ATb"} \NormalTok{=}\StringTok{ }\NormalTok{ATb)}
\NormalTok{results}
\end{Highlighting}
\end{Shaded}

\begin{verbatim}
## $ATA
##      [,1] [,2]
## [1,]    4    8
## [2,]    8   26
## 
## $ATb
##      [,1]
## [1,]   36
## [2,]  112
\end{verbatim}

\subsection{\texorpdfstring{Solve for \(\hat{x}\) in R using the above
computed
matrices}{Solve for \textbackslash{}hat\{x\} in R using the above computed matrices}}\label{solve-for-hatx-in-r-using-the-above-computed-matrices}

\begin{Shaded}
\begin{Highlighting}[]
\NormalTok{x <-}\StringTok{ }\KeywordTok{solve}\NormalTok{(ATA) %*%}\StringTok{ }\NormalTok{ATb}
\NormalTok{x}
\end{Highlighting}
\end{Shaded}

\begin{verbatim}
##      [,1]
## [1,]    1
## [2,]    4
\end{verbatim}

\subsection{What is the squared error of this
solution?}\label{what-is-the-squared-error-of-this-solution}

\begin{Shaded}
\begin{Highlighting}[]
\NormalTok{p <-}\StringTok{ }\NormalTok{A %*%}\StringTok{ }\NormalTok{x}
\CommentTok{#b = p + e or e = p - b which we can substitute in our given values. }
\NormalTok{e <-}\StringTok{ }\NormalTok{p -}\StringTok{ }\NormalTok{b}
\CommentTok{# we then sum the square of errors. }
\NormalTok{e2 <-}\StringTok{ }\KeywordTok{sum}\NormalTok{(e^}\DecValTok{2}\NormalTok{)}
\NormalTok{e2}
\end{Highlighting}
\end{Shaded}

\begin{verbatim}
## [1] 44
\end{verbatim}

\subsection{Find the exact solution with p instead of
b}\label{find-the-exact-solution-with-p-instead-of-b}

\begin{Shaded}
\begin{Highlighting}[]
\KeywordTok{options}\NormalTok{(}\DataTypeTok{scipen =} \DecValTok{999}\NormalTok{)}
\NormalTok{p   <-}\StringTok{ }\KeywordTok{matrix}\NormalTok{(}\KeywordTok{c}\NormalTok{(}\DecValTok{1}\NormalTok{,}\DecValTok{5}\NormalTok{,}\DecValTok{13}\NormalTok{,}\DecValTok{17}\NormalTok{))}
\NormalTok{ATp <-}\StringTok{ }\KeywordTok{t}\NormalTok{(A) %*%}\StringTok{ }\NormalTok{p}
\NormalTok{xp  <-}\StringTok{ }\KeywordTok{solve}\NormalTok{(ATA) %*%}\StringTok{ }\NormalTok{ATp}
\NormalTok{p2  <-}\StringTok{ }\NormalTok{A %*%}\StringTok{ }\NormalTok{xp}
\NormalTok{e   <-}\StringTok{ }\NormalTok{p2-p}
\NormalTok{e}
\end{Highlighting}
\end{Shaded}

\begin{verbatim}
##                          [,1]
## [1,] 0.0000000000000000000000
## [2,] 0.0000000000000008881784
## [3,] 0.0000000000000035527137
## [4,] 0.0000000000000035527137
\end{verbatim}

Essentially, the error vector e is = 0.

\begin{Shaded}
\begin{Highlighting}[]
\NormalTok{e2p <-}\StringTok{ }\KeywordTok{sum}\NormalTok{(e^}\DecValTok{2}\NormalTok{)}
\NormalTok{e2p}
\end{Highlighting}
\end{Shaded}

\begin{verbatim}
## [1] 0.00000000000000000000000000002603241
\end{verbatim}

Show that the error \(e = b - p = [-1;3;-5;3]\).

\begin{Shaded}
\begin{Highlighting}[]
\NormalTok{b -}\StringTok{ }\NormalTok{p}
\end{Highlighting}
\end{Shaded}

\begin{verbatim}
##      [,1]
## [1,]   -1
## [2,]    3
## [3,]   -5
## [4,]    3
\end{verbatim}

Show that the error \(e\) is orthogonal to \(p\) and to each of the
columns of \(A\).

As per the week 5 handout - We know that when two vectors are
orthogonal, their dot product is zero.

\begin{Shaded}
\begin{Highlighting}[]
\NormalTok{e*p}
\end{Highlighting}
\end{Shaded}

\begin{verbatim}
##                         [,1]
## [1,] 0.000000000000000000000
## [2,] 0.000000000000004440892
## [3,] 0.000000000000046185278
## [4,] 0.000000000000060396133
\end{verbatim}

\begin{Shaded}
\begin{Highlighting}[]
\KeywordTok{sum}\NormalTok{(e*A[,}\DecValTok{1}\NormalTok{])}
\end{Highlighting}
\end{Shaded}

\begin{verbatim}
## [1] 0.000000000000007993606
\end{verbatim}

\section{Problem Set 2}\label{problem-set-2}

Write an R markdown script that takes in the auto-mpg data, extracts an
A matrix from the first 4 columns and b vector from the fifth (mpg)
column.

Apparently, an added column of 1 is necessary to obtain an intercept.

\begin{Shaded}
\begin{Highlighting}[]
\NormalTok{x <-}\StringTok{ }\KeywordTok{as.matrix}\NormalTok{(}\KeywordTok{read.table}\NormalTok{(}\StringTok{"https://raw.githubusercontent.com/ChristopheHunt/MSDA---Coursework/master/Data%20605/Assignment%205/auto-mpg.data"}\NormalTok{))}

\NormalTok{A <-}\StringTok{ }\KeywordTok{as.matrix}\NormalTok{(}\KeywordTok{cbind}\NormalTok{(x[,}\DecValTok{1}\NormalTok{:}\DecValTok{4}\NormalTok{],}\DecValTok{1}\NormalTok{))}
\NormalTok{b <-}\StringTok{ }\KeywordTok{as.matrix}\NormalTok{(x[,}\DecValTok{5}\NormalTok{])}
\end{Highlighting}
\end{Shaded}

Using the least squares approach, your code should compute the best
fitting solution

\begin{Shaded}
\begin{Highlighting}[]
\NormalTok{ATA <-}\StringTok{ }\KeywordTok{t}\NormalTok{(A) %*%}\StringTok{ }\NormalTok{A}
\NormalTok{ATb <-}\StringTok{ }\KeywordTok{t}\NormalTok{(A) %*%}\StringTok{ }\NormalTok{b}
\NormalTok{results <-}\StringTok{ }\KeywordTok{list}\NormalTok{(}\StringTok{"ATA"} \NormalTok{=}\StringTok{ }\NormalTok{ATA, }\StringTok{"ATb"} \NormalTok{=}\StringTok{ }\NormalTok{ATb)}
\NormalTok{results}
\end{Highlighting}
\end{Shaded}

\begin{verbatim}
## $ATA
##             V1          V2         V3         V4          
## V1  19097634.2   9374647.0  259345480  1123011.9   76209.5
## V2   9374647.0   4857524.0  132989885   607832.3   40952.0
## V3 259345480.0 132989885.0 3757575489 17758103.6 1167213.0
## V4   1123011.9    607832.3   17758104    97656.9    6092.2
##        76209.5     40952.0    1167213     6092.2     392.0
## 
## $ATb
##          [,1]
## V1  1529685.9
## V2   868718.8
## V3 25209061.4
## V4   146401.4
##        9190.8
\end{verbatim}

\subsection{\texorpdfstring{Solve for \(\hat{x}\) in R using the above
computed
matrices}{Solve for \textbackslash{}hat\{x\} in R using the above computed matrices}}\label{solve-for-hatx-in-r-using-the-above-computed-matrices-1}

\begin{Shaded}
\begin{Highlighting}[]
\NormalTok{x <-}\StringTok{ }\KeywordTok{solve}\NormalTok{(ATA) %*%}\StringTok{ }\NormalTok{ATb}
\NormalTok{x}
\end{Highlighting}
\end{Shaded}

\begin{verbatim}
##            [,1]
## V1 -0.006000871
## V2 -0.043607731
## V3 -0.005280508
## V4 -0.023147999
##    45.251139699
\end{verbatim}

The least squares model using this method is:

\[mpg = -0.006displacement + -0.04361horsepower + -0.00528weight + -0.02315acceleration 
 + 45.25114\]

Finally, calculate the fitting error between the predicted mpg of youur
model and actual mpg.

\subsection{What is the squared error of this
solution?}\label{what-is-the-squared-error-of-this-solution-1}

\begin{Shaded}
\begin{Highlighting}[]
\NormalTok{p <-}\StringTok{ }\NormalTok{A %*%}\StringTok{ }\NormalTok{x}
\CommentTok{#b = p + e or e = p - b which we can substitute in our given values. }
\NormalTok{e <-}\StringTok{ }\NormalTok{p -}\StringTok{ }\NormalTok{b}
\CommentTok{# we then sum the square of errors. }
\NormalTok{e2 <-}\StringTok{ }\KeywordTok{sum}\NormalTok{(e^}\DecValTok{2}\NormalTok{)}
\NormalTok{e2}
\end{Highlighting}
\end{Shaded}

\begin{verbatim}
## [1] 6979.413
\end{verbatim}


\end{document}
