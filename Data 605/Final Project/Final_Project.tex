\documentclass[]{article}
\usepackage{lmodern}
\usepackage{amssymb,amsmath}
\usepackage{ifxetex,ifluatex}
\usepackage{fixltx2e} % provides \textsubscript
\ifnum 0\ifxetex 1\fi\ifluatex 1\fi=0 % if pdftex
  \usepackage[T1]{fontenc}
  \usepackage[utf8]{inputenc}
\else % if luatex or xelatex
  \ifxetex
    \usepackage{mathspec}
  \else
    \usepackage{fontspec}
  \fi
  \defaultfontfeatures{Ligatures=TeX,Scale=MatchLowercase}
\fi
% use upquote if available, for straight quotes in verbatim environments
\IfFileExists{upquote.sty}{\usepackage{upquote}}{}
% use microtype if available
\IfFileExists{microtype.sty}{%
\usepackage{microtype}
\UseMicrotypeSet[protrusion]{basicmath} % disable protrusion for tt fonts
}{}
\usepackage[margin=1in]{geometry}
\usepackage{hyperref}
\hypersetup{unicode=true,
            pdftitle={Final Project},
            pdfauthor={Christophe Hunt},
            pdfborder={0 0 0},
            breaklinks=true}
\urlstyle{same}  % don't use monospace font for urls
\usepackage{graphicx,grffile}
\makeatletter
\def\maxwidth{\ifdim\Gin@nat@width>\linewidth\linewidth\else\Gin@nat@width\fi}
\def\maxheight{\ifdim\Gin@nat@height>\textheight\textheight\else\Gin@nat@height\fi}
\makeatother
% Scale images if necessary, so that they will not overflow the page
% margins by default, and it is still possible to overwrite the defaults
% using explicit options in \includegraphics[width, height, ...]{}
\setkeys{Gin}{width=\maxwidth,height=\maxheight,keepaspectratio}
\IfFileExists{parskip.sty}{%
\usepackage{parskip}
}{% else
\setlength{\parindent}{0pt}
\setlength{\parskip}{6pt plus 2pt minus 1pt}
}
\setlength{\emergencystretch}{3em}  % prevent overfull lines
\providecommand{\tightlist}{%
  \setlength{\itemsep}{0pt}\setlength{\parskip}{0pt}}
\setcounter{secnumdepth}{5}
% Redefines (sub)paragraphs to behave more like sections
\ifx\paragraph\undefined\else
\let\oldparagraph\paragraph
\renewcommand{\paragraph}[1]{\oldparagraph{#1}\mbox{}}
\fi
\ifx\subparagraph\undefined\else
\let\oldsubparagraph\subparagraph
\renewcommand{\subparagraph}[1]{\oldsubparagraph{#1}\mbox{}}
\fi

%%% Use protect on footnotes to avoid problems with footnotes in titles
\let\rmarkdownfootnote\footnote%
\def\footnote{\protect\rmarkdownfootnote}

%%% Change title format to be more compact
\usepackage{titling}

% Create subtitle command for use in maketitle
\newcommand{\subtitle}[1]{
  \posttitle{
    \begin{center}\large#1\end{center}
    }
}

\setlength{\droptitle}{-2em}
  \title{Final Project}
  \pretitle{\vspace{\droptitle}\centering\huge}
  \posttitle{\par}
  \author{Christophe Hunt}
  \preauthor{\centering\large\emph}
  \postauthor{\par}
  \predate{\centering\large\emph}
  \postdate{\par}
  \date{May 13, 2017}

\usepackage{relsize}
\usepackage{setspace}
\usepackage{amsmath,amsfonts,amsthm}
\usepackage[sfdefault]{roboto}
\usepackage[T1]{fontenc}
\usepackage{float}
\usepackage{multirow}
\usepackage{mathtools}
\usepackage{tikz}

\begin{document}
\maketitle

{
\setcounter{tocdepth}{2}
\tableofcontents
}
Pick one of the quantitative independent variables from the training
data set (train.csv) , and define that variable as X.

Pick SalePrice as the dependent variable, and define it as Y for the
next analysis.

\section{Probability}\label{probability}

Calculate as a minimum the below probabilities a through c.

Assume the small letter ``x'' is estimated as the 4th quartile of the X
variable, and the small letter ``y'' is estimated as the 2nd quartile of
the Y variable. Interpret the meaning of all probabilities.

\begin{enumerate}
\def\labelenumi{\alph{enumi}.}
\item
\begin{verbatim}
$P(X>x | Y>y)$  
\end{verbatim}
\item
  \(P(X>x, Y>y)\)
\item
  \(P(X<x | Y>y)\)
\end{enumerate}

Does splitting the training data in this fashion make them independent?

In other words, does \(P(X|Y)=P(X)P(Y))\)?

Check mathematically, and then evaluate by running a Chi Square test for
association.

You might have to research this.

\section{Descriptive and Inferential
Statistics.}\label{descriptive-and-inferential-statistics.}

Provide univariate descriptive statistics and appropriate plots for both
variables.

Provide a scatterplot of X and Y.

Transform both variables simultaneously using Box-Cox transformations.\\
You might have to research this.

Using the transformed variables, run a correlation analysis and
interpret.

Test the hypothesis that the correlation between these variables is 0
and provide a 99\% confidence interval.

Discuss the meaning of your analysis.

\section{Linear Algebra and
Correlation.}\label{linear-algebra-and-correlation.}

Invert your correlation matrix. (This is known as the precision matrix
and contains variance inflation factors on the diagonal.) Multiply the
correlation matrix by the precision matrix, and then multiply the
precision matrix by the correlation matrix.

\section{Calculus-Based Probability \&
Statistics}\label{calculus-based-probability-statistics}

Many times, it makes sense to fit a closed form distribution to data.
For your non-transformed independent variable, location shift it so that
the minimum value is above zero.

Then load the MASS package and run fitdistr to fit a density function of
your choice. (See
\url{https://stat.ethz.ch/R-manual/R-devel/library/MASS/html/fitdistr.html}
).

Find the optimal value of the parameters for this distribution, and then
take 1000 samples from this distribution (e.g., rexp(1000, ???) for an
exponential).

Plot a histogram and compare it with a histogram of your non-transformed
original variable.

\section{Modeling}\label{modeling}

Build some type of regression model and submit your model to the
competition board.\\
Provide your complete model summary and results with analysis.

Report your Kaggle.com user name and score.

Multiply the correlation matrix by the precision matrix, and then
multiply the precision matrix by the correlation matrix.


\end{document}
