\documentclass[]{article}
\usepackage{lmodern}
\usepackage{amssymb,amsmath}
\usepackage{ifxetex,ifluatex}
\usepackage{fixltx2e} % provides \textsubscript
\ifnum 0\ifxetex 1\fi\ifluatex 1\fi=0 % if pdftex
  \usepackage[T1]{fontenc}
  \usepackage[utf8]{inputenc}
\else % if luatex or xelatex
  \ifxetex
    \usepackage{mathspec}
  \else
    \usepackage{fontspec}
  \fi
  \defaultfontfeatures{Ligatures=TeX,Scale=MatchLowercase}
\fi
% use upquote if available, for straight quotes in verbatim environments
\IfFileExists{upquote.sty}{\usepackage{upquote}}{}
% use microtype if available
\IfFileExists{microtype.sty}{%
\usepackage{microtype}
\UseMicrotypeSet[protrusion]{basicmath} % disable protrusion for tt fonts
}{}
\usepackage[margin=1in]{geometry}
\usepackage{hyperref}
\hypersetup{unicode=true,
            pdftitle={Homework 7},
            pdfauthor={Christophe Hunt},
            pdfborder={0 0 0},
            breaklinks=true}
\urlstyle{same}  % don't use monospace font for urls
\usepackage{color}
\usepackage{fancyvrb}
\newcommand{\VerbBar}{|}
\newcommand{\VERB}{\Verb[commandchars=\\\{\}]}
\DefineVerbatimEnvironment{Highlighting}{Verbatim}{commandchars=\\\{\}}
% Add ',fontsize=\small' for more characters per line
\usepackage{framed}
\definecolor{shadecolor}{RGB}{248,248,248}
\newenvironment{Shaded}{\begin{snugshade}}{\end{snugshade}}
\newcommand{\KeywordTok}[1]{\textcolor[rgb]{0.13,0.29,0.53}{\textbf{{#1}}}}
\newcommand{\DataTypeTok}[1]{\textcolor[rgb]{0.13,0.29,0.53}{{#1}}}
\newcommand{\DecValTok}[1]{\textcolor[rgb]{0.00,0.00,0.81}{{#1}}}
\newcommand{\BaseNTok}[1]{\textcolor[rgb]{0.00,0.00,0.81}{{#1}}}
\newcommand{\FloatTok}[1]{\textcolor[rgb]{0.00,0.00,0.81}{{#1}}}
\newcommand{\ConstantTok}[1]{\textcolor[rgb]{0.00,0.00,0.00}{{#1}}}
\newcommand{\CharTok}[1]{\textcolor[rgb]{0.31,0.60,0.02}{{#1}}}
\newcommand{\SpecialCharTok}[1]{\textcolor[rgb]{0.00,0.00,0.00}{{#1}}}
\newcommand{\StringTok}[1]{\textcolor[rgb]{0.31,0.60,0.02}{{#1}}}
\newcommand{\VerbatimStringTok}[1]{\textcolor[rgb]{0.31,0.60,0.02}{{#1}}}
\newcommand{\SpecialStringTok}[1]{\textcolor[rgb]{0.31,0.60,0.02}{{#1}}}
\newcommand{\ImportTok}[1]{{#1}}
\newcommand{\CommentTok}[1]{\textcolor[rgb]{0.56,0.35,0.01}{\textit{{#1}}}}
\newcommand{\DocumentationTok}[1]{\textcolor[rgb]{0.56,0.35,0.01}{\textbf{\textit{{#1}}}}}
\newcommand{\AnnotationTok}[1]{\textcolor[rgb]{0.56,0.35,0.01}{\textbf{\textit{{#1}}}}}
\newcommand{\CommentVarTok}[1]{\textcolor[rgb]{0.56,0.35,0.01}{\textbf{\textit{{#1}}}}}
\newcommand{\OtherTok}[1]{\textcolor[rgb]{0.56,0.35,0.01}{{#1}}}
\newcommand{\FunctionTok}[1]{\textcolor[rgb]{0.00,0.00,0.00}{{#1}}}
\newcommand{\VariableTok}[1]{\textcolor[rgb]{0.00,0.00,0.00}{{#1}}}
\newcommand{\ControlFlowTok}[1]{\textcolor[rgb]{0.13,0.29,0.53}{\textbf{{#1}}}}
\newcommand{\OperatorTok}[1]{\textcolor[rgb]{0.81,0.36,0.00}{\textbf{{#1}}}}
\newcommand{\BuiltInTok}[1]{{#1}}
\newcommand{\ExtensionTok}[1]{{#1}}
\newcommand{\PreprocessorTok}[1]{\textcolor[rgb]{0.56,0.35,0.01}{\textit{{#1}}}}
\newcommand{\AttributeTok}[1]{\textcolor[rgb]{0.77,0.63,0.00}{{#1}}}
\newcommand{\RegionMarkerTok}[1]{{#1}}
\newcommand{\InformationTok}[1]{\textcolor[rgb]{0.56,0.35,0.01}{\textbf{\textit{{#1}}}}}
\newcommand{\WarningTok}[1]{\textcolor[rgb]{0.56,0.35,0.01}{\textbf{\textit{{#1}}}}}
\newcommand{\AlertTok}[1]{\textcolor[rgb]{0.94,0.16,0.16}{{#1}}}
\newcommand{\ErrorTok}[1]{\textcolor[rgb]{0.64,0.00,0.00}{\textbf{{#1}}}}
\newcommand{\NormalTok}[1]{{#1}}
\usepackage{graphicx,grffile}
\makeatletter
\def\maxwidth{\ifdim\Gin@nat@width>\linewidth\linewidth\else\Gin@nat@width\fi}
\def\maxheight{\ifdim\Gin@nat@height>\textheight\textheight\else\Gin@nat@height\fi}
\makeatother
% Scale images if necessary, so that they will not overflow the page
% margins by default, and it is still possible to overwrite the defaults
% using explicit options in \includegraphics[width, height, ...]{}
\setkeys{Gin}{width=\maxwidth,height=\maxheight,keepaspectratio}
\IfFileExists{parskip.sty}{%
\usepackage{parskip}
}{% else
\setlength{\parindent}{0pt}
\setlength{\parskip}{6pt plus 2pt minus 1pt}
}
\setlength{\emergencystretch}{3em}  % prevent overfull lines
\providecommand{\tightlist}{%
  \setlength{\itemsep}{0pt}\setlength{\parskip}{0pt}}
\setcounter{secnumdepth}{5}
% Redefines (sub)paragraphs to behave more like sections
\ifx\paragraph\undefined\else
\let\oldparagraph\paragraph
\renewcommand{\paragraph}[1]{\oldparagraph{#1}\mbox{}}
\fi
\ifx\subparagraph\undefined\else
\let\oldsubparagraph\subparagraph
\renewcommand{\subparagraph}[1]{\oldsubparagraph{#1}\mbox{}}
\fi

%%% Use protect on footnotes to avoid problems with footnotes in titles
\let\rmarkdownfootnote\footnote%
\def\footnote{\protect\rmarkdownfootnote}

%%% Change title format to be more compact
\usepackage{titling}

% Create subtitle command for use in maketitle
\newcommand{\subtitle}[1]{
  \posttitle{
    \begin{center}\large#1\end{center}
    }
}

\setlength{\droptitle}{-2em}
  \title{Homework 7}
  \pretitle{\vspace{\droptitle}\centering\huge}
  \posttitle{\par}
  \author{Christophe Hunt}
  \preauthor{\centering\large\emph}
  \postauthor{\par}
  \predate{\centering\large\emph}
  \postdate{\par}
  \date{March 18, 2017}

\usepackage{relsize}
\usepackage{setspace}
\usepackage{amsmath,amsfonts,amsthm}
\usepackage[sfdefault]{roboto}
\usepackage[T1]{fontenc}
\usepackage{float}
\usepackage{multirow}
\usepackage{mathtools}

\begin{document}
\maketitle

{
\setcounter{tocdepth}{2}
\tableofcontents
}
\section{Problem Set 1}\label{problem-set-1}

This week, you'll have only one programming assignment. Please write a
function to compute the expected value and standard deviation of an
array of values. Compare your results with that of R's mean and std
functions. Please document your work in an RMarkdown file and ensure
that you have good comments to help the reader follow your work.

\begin{Shaded}
\begin{Highlighting}[]
\CommentTok{# vectors of different length}

\NormalTok{V1 <-}\StringTok{ }\KeywordTok{c}\NormalTok{(}\DecValTok{5}\NormalTok{,}\DecValTok{9}\NormalTok{,}\DecValTok{3}\NormalTok{,}\DecValTok{23}\NormalTok{)}
\NormalTok{V2 <-}\StringTok{ }\KeywordTok{c}\NormalTok{(}\DecValTok{10}\NormalTok{,}\DecValTok{11}\NormalTok{,}\DecValTok{12}\NormalTok{,}\DecValTok{13}\NormalTok{,}\DecValTok{14}\NormalTok{,}\DecValTok{15}\NormalTok{,}\DecValTok{10}\NormalTok{,}\DecValTok{13}\NormalTok{)}
\NormalTok{V3 <-}\StringTok{ }\KeywordTok{c}\NormalTok{(}\DecValTok{1}\NormalTok{:}\DecValTok{20}\NormalTok{)}

\CommentTok{# vectors added to array}
\NormalTok{array_x <-}\StringTok{ }\KeywordTok{array}\NormalTok{(}\KeywordTok{c}\NormalTok{(V1,V2, V3))}

\CommentTok{# Expected results is the mean sum/count}

\NormalTok{expected_results <-}\StringTok{ }\NormalTok{function(x)\{}
  \KeywordTok{return}\NormalTok{(}\KeywordTok{sum}\NormalTok{(x)/}\KeywordTok{dim}\NormalTok{(x)[}\DecValTok{1}\NormalTok{])}
\NormalTok{\}}

\CommentTok{# Standard deviation of the sample population is the value less the mean, then squared, }
\CommentTok{#the sum of that divided by the length less 1 (adjustment made for sample population),}
\CommentTok{#then the square root of that final result}

\NormalTok{stand_dv_samp <-}\StringTok{ }\NormalTok{function(x)\{}
  \KeywordTok{return}\NormalTok{(}\KeywordTok{sqrt}\NormalTok{(}\KeywordTok{sum}\NormalTok{((x -}\StringTok{ }\KeywordTok{expected_results}\NormalTok{(x))^}\DecValTok{2}\NormalTok{)/(}\KeywordTok{dim}\NormalTok{(x)[}\DecValTok{1}\NormalTok{] -}\StringTok{ }\DecValTok{1}\NormalTok{)))}
\NormalTok{\}}
\end{Highlighting}
\end{Shaded}

\newpage 

\begin{Shaded}
\begin{Highlighting}[]
\KeywordTok{list}\NormalTok{(}\StringTok{"user defined mean function"} \NormalTok{=}\StringTok{ }\KeywordTok{expected_results}\NormalTok{(array_x),}
     \StringTok{"base R mean function"} \NormalTok{=}\StringTok{ }\KeywordTok{mean}\NormalTok{(array_x))}
\end{Highlighting}
\end{Shaded}

\begin{verbatim}
## $`user defined mean function`
## [1] 10.875
## 
## $`base R mean function`
## [1] 10.875
\end{verbatim}

\begin{Shaded}
\begin{Highlighting}[]
\KeywordTok{list}\NormalTok{(}\StringTok{"user defined sd function"} \NormalTok{=}\StringTok{ }\KeywordTok{stand_dv_samp}\NormalTok{(array_x), }
     \StringTok{"base R sd function"} \NormalTok{=}\StringTok{ }\KeywordTok{sd}\NormalTok{(array_x))}
\end{Highlighting}
\end{Shaded}

\begin{verbatim}
## $`user defined sd function`
## [1] 5.545995
## 
## $`base R sd function`
## [1] 5.545995
\end{verbatim}

Now, consider that instead of being able to neatly fit the values in
memory in an array, you have an infinite stream of numbers coming by.
How would you estimate the mean and standard deviation of such a stream?
Your function should be able to return the current estimate of the mean
and standard deviation at any time it is asked. Your program should
maintain these current estimates and return them back at any invocation
of these functions. (Hint: You can maintain a rolling estimate of the
mean and standard deviation and allow these to slowly change over time
as you see more and more new values).

\begin{Shaded}
\begin{Highlighting}[]
\CommentTok{# vectors of different length}

\NormalTok{V1 <-}\StringTok{ }\KeywordTok{c}\NormalTok{(}\DecValTok{5}\NormalTok{,}\DecValTok{9}\NormalTok{,}\DecValTok{3}\NormalTok{,}\DecValTok{23}\NormalTok{)}
\NormalTok{V2 <-}\StringTok{ }\KeywordTok{c}\NormalTok{(}\DecValTok{10}\NormalTok{,}\DecValTok{11}\NormalTok{,}\DecValTok{12}\NormalTok{,}\DecValTok{13}\NormalTok{,}\DecValTok{14}\NormalTok{,}\DecValTok{15}\NormalTok{,}\DecValTok{10}\NormalTok{,}\DecValTok{13}\NormalTok{)}
\NormalTok{V3 <-}\StringTok{ }\KeywordTok{c}\NormalTok{(}\DecValTok{1}\NormalTok{:}\DecValTok{20}\NormalTok{)}

\CommentTok{# vectors added to array}
\NormalTok{array_x <-}\StringTok{ }\KeywordTok{array}\NormalTok{(}\KeywordTok{c}\NormalTok{(V1,V2, V3))}

\CommentTok{# we create our intial list of values to interate from}
\NormalTok{results <-}\StringTok{ }\KeywordTok{list}\NormalTok{(}\StringTok{"samp_size"} \NormalTok{=}\StringTok{ }\KeywordTok{dim}\NormalTok{(array_x)[}\DecValTok{1}\NormalTok{], }
                \StringTok{"sum"} \NormalTok{=}\StringTok{ }\KeywordTok{sum}\NormalTok{(array_x), }
                \StringTok{"sum_sqrs"} \NormalTok{=}\StringTok{ }\KeywordTok{sum}\NormalTok{(array_x^}\DecValTok{2}\NormalTok{), }
                \StringTok{"last_mean"} \NormalTok{=}\StringTok{ }\KeywordTok{mean}\NormalTok{(array_x),}
                \StringTok{"last_sd"} \NormalTok{=}\StringTok{ }\KeywordTok{sd}\NormalTok{(array_x))}
\NormalTok{results}
\end{Highlighting}
\end{Shaded}

\begin{verbatim}
## $samp_size
## [1] 32
## 
## $sum
## [1] 348
## 
## $sum_sqrs
## [1] 4738
## 
## $last_mean
## [1] 10.875
## 
## $last_sd
## [1] 5.545995
\end{verbatim}

\begin{Shaded}
\begin{Highlighting}[]
\NormalTok{avg_sd_stream <-}\StringTok{ }\NormalTok{function(results, x)\{}
                 \NormalTok{results$samp_size <-}\StringTok{ }\NormalTok{results$samp_size +}\StringTok{ }\KeywordTok{length}\NormalTok{(x)}
                 \NormalTok{results$sum <-}\StringTok{ }\NormalTok{results$sum +}\StringTok{ }\KeywordTok{sum}\NormalTok{(x)}
                 \NormalTok{results$last_mean <-}\StringTok{ }\NormalTok{(results$sum/}\StringTok{ }\NormalTok{results$samp_size) }
                 \NormalTok{results$sum_sqrs <-}\StringTok{ }\NormalTok{results$sum_sqrs +}\StringTok{ }\KeywordTok{sum}\NormalTok{(x^}\DecValTok{2}\NormalTok{)}
                 \NormalTok{sigma_sq <-}\StringTok{ }\NormalTok{(results$sum_sqrs /}\StringTok{ }
\StringTok{                                }\NormalTok{results$samp_size) -}\StringTok{ }
\StringTok{                             }\NormalTok{(results$last_mean^}\DecValTok{2}\NormalTok{)}
                 \NormalTok{results$last_sd <-}\StringTok{ }\KeywordTok{sqrt}\NormalTok{(sigma_sq) *}\StringTok{ }
\StringTok{                                    }\KeywordTok{sqrt}\NormalTok{(results$samp_size/}
\StringTok{                                           }\NormalTok{(results$samp_size}\DecValTok{-1}\NormalTok{))}
                 \KeywordTok{return}\NormalTok{(results) }
\NormalTok{\}}

\NormalTok{results <-}\StringTok{ }\KeywordTok{avg_sd_stream}\NormalTok{(results, }\KeywordTok{c}\NormalTok{(}\DecValTok{1}\NormalTok{,}\DecValTok{2}\NormalTok{,}\DecValTok{4}\NormalTok{,}\DecValTok{5}\NormalTok{,}\DecValTok{6}\NormalTok{,}\DecValTok{7}\NormalTok{))}
\NormalTok{results}
\end{Highlighting}
\end{Shaded}

\begin{verbatim}
## $samp_size
## [1] 38
## 
## $sum
## [1] 373
## 
## $sum_sqrs
## [1] 4869
## 
## $last_mean
## [1] 9.815789
## 
## $last_sd
## [1] 5.713215
\end{verbatim}

\begin{Shaded}
\begin{Highlighting}[]
\NormalTok{results <-}\StringTok{ }\KeywordTok{avg_sd_stream}\NormalTok{(results, }\KeywordTok{c}\NormalTok{(}\DecValTok{20}\NormalTok{, }\DecValTok{30}\NormalTok{, }\DecValTok{40}\NormalTok{, }\DecValTok{50}\NormalTok{))}
\NormalTok{results}
\end{Highlighting}
\end{Shaded}

\begin{verbatim}
## $samp_size
## [1] 42
## 
## $sum
## [1] 513
## 
## $sum_sqrs
## [1] 10269
## 
## $last_mean
## [1] 12.21429
## 
## $last_sd
## [1] 9.881087
\end{verbatim}

\newpage

Now to check out last run of the avg\_sd\_stream function. I duplicate
the full results and re-run using base R functions to see if we are
achieiving the same results.

\begin{Shaded}
\begin{Highlighting}[]
\NormalTok{V4 <-}\StringTok{ }\KeywordTok{c}\NormalTok{(}\DecValTok{1}\NormalTok{,}\DecValTok{2}\NormalTok{,}\DecValTok{4}\NormalTok{,}\DecValTok{5}\NormalTok{,}\DecValTok{6}\NormalTok{,}\DecValTok{7}\NormalTok{)}
\NormalTok{V5 <-}\StringTok{ }\KeywordTok{c}\NormalTok{(}\DecValTok{20}\NormalTok{, }\DecValTok{30}\NormalTok{, }\DecValTok{40}\NormalTok{, }\DecValTok{50}\NormalTok{)}
\CommentTok{# vectors added to array}
\NormalTok{array_x <-}\StringTok{ }\KeywordTok{array}\NormalTok{(}\KeywordTok{c}\NormalTok{(V1,V2, V3, V4, V5))}

\CommentTok{# we create our intial list of values to interate from}
\NormalTok{results <-}\StringTok{ }\KeywordTok{list}\NormalTok{(}\StringTok{"samp_size"} \NormalTok{=}\StringTok{ }\KeywordTok{dim}\NormalTok{(array_x)[}\DecValTok{1}\NormalTok{], }
                \StringTok{"sum"} \NormalTok{=}\StringTok{ }\KeywordTok{sum}\NormalTok{(array_x), }
                \StringTok{"sum_sqrs"} \NormalTok{=}\StringTok{ }\KeywordTok{sum}\NormalTok{(array_x^}\DecValTok{2}\NormalTok{), }
                \StringTok{"last_mean"} \NormalTok{=}\StringTok{ }\KeywordTok{mean}\NormalTok{(array_x),}
                \StringTok{"last_sd"} \NormalTok{=}\StringTok{ }\KeywordTok{sd}\NormalTok{(array_x))}
\NormalTok{results}
\end{Highlighting}
\end{Shaded}

\begin{verbatim}
## $samp_size
## [1] 42
## 
## $sum
## [1] 513
## 
## $sum_sqrs
## [1] 10269
## 
## $last_mean
## [1] 12.21429
## 
## $last_sd
## [1] 9.881087
\end{verbatim}


\end{document}
